The SMODERP2D episodic rainfall-runoff/erosion model  based on a 1D profile
version, in which the surface runoff and the erosion were typically calculated
in several 1D profiles representing the main flow path in the hillslope
\cite{Dostal2000}.  The current generation of the SMODERP2D model is pixel
distributed, and is implemented in python in order to be compatible with most
GIS software. The development is presented on the github platform at
\href{https://github.com/storm-fsv-cvut/smoderp2d}{github.com/storm-fsv-cvut/smoderp2d}.

The SMODERP2D model is primarily designed for surface runoff and erosion
computation. The surface flow routing in the model is based on the digital
elevation model (DEM). For simplicity, DEM also controls the spatial
discretization of the model. 


\subsection{Model history and development}
The model has been developed since the end of the 1980´s at the Department of
Irrigation, Drainage and Land Scape Engineering, Faculty of Civil Engineering,
Czech Technical University in Prague. The first version of 1D model, designed
in the FORTRAN programming language, was developed in 1989. It included
processes influencing surface runoff and erosion. Two basically independent
submodels: (i) submodel 1 used for the calculation of admissible slope length
and designing of soil conservation measures, (ii) submodel 2 designed for the
calculation of runoff characteristics in order to implement technical
anti-erosion measures. Other versions of the model, launched particularly
between 1996 and 2001, further developed operating systems. However, these
versions did not considerably interfere with parameters and relations inside
the model.  

Compared to the previously introduced versions, a change related to runoff
parameters in the kinematic wave equation depending on soil types in accordance
with the Czech system of soil classification (CTCSS) occurred. This system uses
for the determination of basic soil types the content of particles up to 0.01
mm of the first size particle category (physical clay). The second modification
was introduced by replacement of the current generation in slopes. Previously,
the model functioned in particular sections of various lengths which had been
initially defined as parts of the slope between two contour lines with
identical vegetation and soil types. At the moment, the new model divides the
aforesaid sections into individual elements of the same length and forms thus
an interphase terminated by testing and comparison to the distributed 2D model. 

The 2D model which was developed in last years is prepared as an extension to
the GIS software such as grass, QGIS, or  ArcGIS. The computational core of the
2D model was elaborated in Python and used in each provider. Additionally,
mathematical description of rill flow and flow in watercourses were implemented
in the 2D model version.


%!TEX ROOT = ../main.tex
Mobile rainfall simulator operated by CTU in Prague was constructed. Its detailed description can be found in \cite{EGUDS}. The essential part of simulator is a folding boom with nine nozzles (Spraying Systems FullJet 40WSQ \citep{Strauss2000})
controlled by electromagnetic valves. The boom consists of a dural steel framed structure and four pairs telescopic legs. Construction can completely standalone or can be folded out directly from the trailer and heaved by a pulley into desired height. In order to prevent the wind from affecting the simulation the supporting construction is covered with the tarpaulin.
The device includes a 1 m3 tank, portable generator, pump producing a constant pressure with minimum delivery 40 l/min and a control unit for managing the rainfall intensity by triggering the electromagnetic valves. The electromagnetic valves trigger the separate three triples of nozzles. Their opening and closing in an arbitrary time interval produces the desired intensity of simulated rainfall. Coupling the nozzles into triples reduces hydraulic shocks in the water distribution system and therefore helps maintaining an uniform spatial distribution of water over the experimental plot.
Four nozzles FullJet 40WSQ are positioned 2.65 m above the ground and 1.2 m apart. Type of the nozzles as well as the position were selected after many calibration tests aimed at reaching as uniform spatial distribution and droplet characteristics as similar to natural raindrops as possible. Final configuration yielded the Christiansen’s uniformity index \cite{Howell2003} of 80\% and more for all rainfall scenarios commonly used in the field simulations.
This configuration and water pressure 0.72 bar leads to the total watered area 2.4 x 10.0 m. It can be used for an arbitrary setup of experimental plots, although limited by a decreasing rainfall intensity towards the edges of the watered area.

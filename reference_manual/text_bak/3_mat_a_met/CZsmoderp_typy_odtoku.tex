%!TEX ROOT = ../mainCZ.tex
%\section{Nejběžnější typy odtoků a jejich implementace do programu} \label{typyodtoku}
Existují tři nejběžnější algoritmy, každý odlišně řešící problematiku odtoku na rastrovém modelu terénu. Prvním z~nich je algoritmus D8 z~roku 1984 \cite{Modely1986}.
Algoritmus je podrobně popsán v~následující kapitole~\ref{subsection:d8}. Jelikož odtok nastává vždy jen do~jedné sousední buňky, dochází často ke~koncentraci vodních sítí a~síť odtoku pak neodpovídá skutečnosti.
Na~druhou stranu je jeho výhodou jeho jednoduchost a~fakt, že~nedochází k~něchtené disperzi, jako je tomu u~druhého algoritmu.

Druhým algoritmem je MD8, neboli Multiple Flow Direction Algorithm.
MD8 algoritmem je voda odváděna do~všech níže položených sousedních buněk a~poměr vody je rozdělen procentuálně podle~sklonu. Ve~výsledku má algoritmus tendenci vytvářet více realistické prostorové sítě toku než~D8 
tím, že nedochází k~takové koncentraci vody v~síti. U~D8 má mnohem větší vliv malý rozdíl výšek mezi sousedními buňkami na~to, která~buňka přijme veškerou vodu.
Použitím MD8 tyto malé výškové rozdíly nemají zdaleka takový vliv, jelikož je množství vody rozděleno poměrem mezi níže položené buňky. Největší nevýhodou tohoto algoritmu je jeho přílišná disperze na~konvergentních svazích v~důsledku odtoku do~všech níže položených buněk.

V~roce 1997 byl navržen algoritmus pojmenovaný D$\infty$ Davidem G. Tarbotonem z~americké univerzity v~Utahu, který~je založen na~osmi trojúhelníkových ploškách.%, viz.~obrázek \ref{fig:TF}. 
Díky tomuto postupu jsou odstraněny limitace v~podobě možnosti pouze osmi směrů odtoku z~buňky. Proto je algoritmus pojmenován D$\infty$, jelikož může nastat nekonečně možností výsledného směru odtoku. 
Tento~přístup umožňuje pouze jeden směr odtoku, rozdělený mezi~jednu nebo~dvě buňky v~závislosti na~směru odtoku.

Poslední dva zmíněné algoritmy MD8 a~D$\infty$ mají oproti~D8 nevýhodu ve~větší časové náročnosti výpočtu. %V~kapitole~\ref{subsection:MD} je popsán algoritmus MD$\infty$, který využívá výhody MD8 algoritmu za~použití trojúhelníkových plošek obdobně jako D$\infty$ algoritmus.
V programu SMODERP 2D jsou použity dva přístupy. Pro plošný odtok uživatel volí mezi jednosměrným odtokem MD$\infty$. Soustředěný odtok v rýhách využívá jen jednosměrného odtoku D8.
%!TEX ROOT = ../mainCZ.tex
\paragraph{Profilové verze - SMODERP 1D}  \mbox{} \\
Vývoj modelu je možné rozdělit do dvou základní etap. V tém první se jednalo o vývoj jednorozměrného modelu, tzv. profilové verze. V druhé pak o rozvoj prostorového řešení (2D).
\subparagraph{1986 - 1996}  \mbox{} \\
První měření vedoucí ke odvození modelu bylo provedeno na sklopném hydraulickém žlabu \citep{Holy1984}. Na základě těchto měření byl stanoven vztah pro výpočet průtoku na základě výšky hladiny. První verze modelu byla vyvinuta v roce 1989. Zahrnovala v sobě procesy ovlivňující povrchový odtok a erozi. V zásadě to byly dva oddělené submodely  (i) submodel pro výpočet přípustné délky  a (ii) submodel pro výpočet odtokových charakteristik. Model pracoval v charakteristických profilech zadávaných formou vodorovné vzdálenosti v konstantní odlehlosti vrstevnic. Počet takto definovaných úseků byl omezen na deset. Časový krok modelu byl fixní 0,2 minuty.

Fyzikální popis prodění vody byl postaven na modifikované kinematické rovnici pro plošný odtok \acs{MKWA} ve tvaru:

$q = a.h^b$

Z měření bylo odvozeno že parametr $a$ je funkcí sklonu zatímco parametr $b$ je pouze funkcí půdního typu. Parametry byly odvozeny pouze pro tři půdní druhy, což odpovídalo tehdejší taxonomii půd.
\subparagraph{2001 - 2006}  \mbox{} \\
Tato verze byla aktualizována v roce 2001 pod označením verze 2.20. Především šlo o vylepšení uživatelského rozhraní. V roce 2001 byla také vytvořena zkušební verze 3.55, ve které byl zahrnut kromě simulace erozní ohroženosti i výpočet odtokových charakteristik. Tato verze byla v roce 2003 aktualizována zcela funkční verzí 4.01. Poslední verzí, která přímo vychází a navazuje na předchozí verze, má označení 5.1 a vznikla v roce 2010. Z výpočtu byl kvůli nedostatku dat pro verifikaci vyřazen výpočet ztráty půdy. Vedle základních odtokových charakteristik jsou uváděny také další charakteristiky (maximální výška hladiny, rychlost proudění a doba kulminace). Tato verze je přechodem mezi navazující verze. Ve verzích 2.20 až 5.1 nebylo nijak zasahováno do výpočetních procesů v modelu.
\subparagraph{2010 - 2012}  \mbox{} \\
Nová verze modelu SMODERP s označením 10.01, byla naprogramována v programovacím jazyku Visual FoxPro. Model verze 10.01 je také fyzikálně založený epizodní jednorozměrný model. Nejedná se již o model částečně dělný, ale o model plně distribuovaný . Dále jsou do modelu zahrnuty rekalibrováné odtokové parametry. V modelu jsou zachovány dříve ověřené fyzikální vztahy. Do verze 10.01 byly zahrnuty následující změny:
\begin{itemize}
\item upuštění od výpočtu ztráty půdy,
\item nově určené odtokové parametry pro jednotlivé kategorie půdních druhů podle Nováka \cite{Nemecek2011},
\item v modelu je možné zadávat libovolně dlouhé na vrstevnicích nezávislé části svahu,
\item model bude pracovat v jednotlivých elementech, jejich velikost bude dána zvoleným charakterem řešeného profilu. Jedná se o zásadní přechod mezi semidistrubuovaným a distribuovaným přístupem,
\item určení přerušení svahu.
\end{itemize}

\paragraph{Prostorové řešení - SMODERP 2D} \mbox{} \\

Od roku 2012 je vyvíjeno prostorové řešení výpočtu povrchového odtoku. V návaznosti na rozvoj \acs{GIS} software a výpočetní techniky vůbec je  dalším logickým krokem vytvořit prostorově distribuovaný model. Myšlenku tohoto dalšího rozvoje modelu stanovil \cite {KavkaDisertace}. Základními úlohami a otázkami v řešení modelu bylo:
\begin{itemize}
\item je možné zjednodušující předpoklady zahrnuté do výpočtu profilové metody aplikovat do prostorově děleného modelu,
\item pro výpočet v jednotlivých buňkách rastru je třeba do dvourozměrného modelu zahrnout algoritmus odtoku resp. přítoku mezi jednotlivými buňkami rastru, 
\item do modelu zavést dynamický výpočtový krok a zajistit numerickou stabilitu výpočtu, 
\item zavedení výpočtu soustředěného odtoku v rýhách, které profilová metoda nijak nezohledňovala,
\item ověřit dostupnost vstupních dat pro prostorové řešení a zajistit jejich zpracování do 2D verze.
\end{itemize}

Zvolené řešení bylo postavené na platformě ESRI GIS. Která používá jako svůj nativní skriptovací jazyk Python. Knihovny ESRI ArcGIS jsou využívány na zpracování vstupních dat a pro finalizaci geoprostorých výstupů. Vstupní data o vegetaci, půdě, DMT a srážce jsou převedeny do rastrové reprezentace v pravidelné mřížce odpovídající DMT. Po zpracování vstupních dat a jejich převedení na rastrovou reprezentaci jsou převedeny do matic a vlastní výpočet již běží nezávisle na palatformě ArCGIS.
Dynamický časový krok a zároveň zajištění stability výpočtu bylo ve výpočtu vyřešeno pomocí \ac{cour}. Prostorová mřížka výpočtu je neměnná, výpočetní časový krok \acs{dT} se v modelu mění úměrně vypočtené rychlosti.
Vzájemné vazby mezi buňkami byli nejprve řešeny pomocí AcgGIS nativního výpočtu jednosměrného odtoku. Později byla do modelu implementována metoda vícesměnného odtoku. Soustředěný odtok v rýhách byl do modelu zaveden jako oddělený podporces, který je spouštěn v buňkách, kde dojde k překročení kritického proudění vody a dochází tak k vymílání půdních částic.  

\subparagraph{do diskuse dát}
\begin{itemize}
\item jiné platformy
\item jádra
\end{itemize}
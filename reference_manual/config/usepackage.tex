%Petr - pouzivam



\usepackage{xspace}
\usepackage{framed}
\usepackage{mathtools}
\usepackage{amsmath}
\usepackage{pdflscape}
\usepackage[top=3cm, left=3.5cm, right=2.5cm, bottom=3cm, headheight=15pt, includeheadfoot]{geometry}%rozměry stránky
\usepackage{textcomp}
\usepackage{natbib}
\usepackage{hyperref}
\hypersetup{
    bookmarks=true,         % show bookmarks bar?
    unicode=false,          % non-Latin characters in Acrobat’s bookmarks
    pdftoolbar=true,        % show Acrobat’s toolbar?
    pdfmenubar=true,        % show Acrobat’s menu?
    pdffitwindow=false,     % window fit to page when opened
    pdfstartview={FitH},    % fits the width of the page to the window
    pdftitle={Smoderp manual},    % title
    pdfauthor={Kavka ...},     % author
    pdfsubject={Subject},   % subject of the document
    pdfcreator={Creator},   % creator of the document
    pdfproducer={Producer}, % producer of the document
    pdfkeywords={keyword1, key2, key3}, % list of keywords
    pdfnewwindow=true,      % links in new PDF window
    colorlinks=false,       % false: boxed links; true: colored links
    linkcolor=red,          % color of internal links (change box color with linkbordercolor)
    citecolor=green,        % color of links to bibliography
    filecolor=magenta,      % color of file links
    urlcolor=cyan           % color of external links
}
\usepackage[printonlyused]{acronym} %rejstik
% \usepackage{acronym} %rejstik
\makeatletter
\AtBeginDocument{%
  \renewcommand*{\AC@hyperlink}[2]{#2}%
}
\makeatother








% kvuli literature v htlatex
% \usepackage[nomain]{glossaries}
% \makeglossaries

%text
\usepackage{subcaption}
\captionsetup{compatibility=false}


\usepackage{listings} % inserting code 
\lstset
{ %Formatting for code in appendix
    language=Python,
    basicstyle=\footnotesize,
%     numbers=left,
%     stepnumber=1,
%     showstringspaces=false,
%     tabsize=1,
%     breaklines=true,
%     breakatwhitespace=true,
}


%tabulky
\usepackage{array}
\usepackage{tabulary}
\usepackage{multirow}
\usepackage{multicol}

%obrazky





% nuti obrazky aby nepretekali do dalsi sekce
\usepackage[section]{placeins}








%Lenka - nepouzivam
\usepackage{longtable}
\usepackage{siunitx}
\usepackage{rotating}
\usepackage[table,xcdraw]{xcolor}
\usepackage{booktabs}
\usepackage{url}
%\usepackage[pdftex,unicode,bookmarksnumbered,raiselinks=true]{hyperref}
% \usepackage{indentfirst}
\usepackage{fancyhdr}
\usepackage[font={footnotesize},labelfont=bf,justification=justified]{caption}
\usepackage{hhline}
\usepackage{colortbl}
\usepackage{array}
\usepackage{graphicx}

\usepackage{titlesec}



% \usepackage[table]{xcolor}
\usepackage{setspace}


% vetsi mezera mezi odstavci
\setlength{\parskip}{0.5em}


\usepackage{forest}    % na dir tree ale obecnejsi uziti
% \usepackage{indentfirst}

% tohle je jen prikaz na delatni popisu rovnic 
% jeho pouziti he v kapitole pouzite vztahy treba
\newcommand{\jj}[2]{
   & \acs{#1} & je \acl{#1}#2 \\
}




% kolek okolo cisla, pouzito v tabulce popisujici toolbox
\newcommand*\circled[1]{\tikz[baseline=(char.base)]{
            \node[shape=circle,draw,inner sep=2pt] (char) {{\scriptsize\sffamily#1}};}}



% Název smodepu 
% 
\newcommand{\smod}{SMODERP2D\xspace}


 
\usepackage{breqn}
 
%%%% upraveni sekce
% \titlespacing*{\section}
% {0pt}{7.5ex}{2.5ex} 
 

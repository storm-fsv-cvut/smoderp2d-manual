% \documentclass[a4paper, 11pt, twoside]{article}
\documentclass[a4paper, 11pt, twoside, draft]{article}
%Petr - pouzivam

%obecne
\usepackage[czech]{babel}
\usepackage[utf8]{inputenc}
\usepackage[IL2]{fontenc}
% \usepackage{charter}



\usepackage{xspace}
\usepackage{framed}
\usepackage{mathtools}
\usepackage{pdflscape}
\usepackage[top=3cm, left=3.5cm, right=2.5cm, bottom=3cm, headheight=15pt, includeheadfoot]{geometry}%rozměry stránky
\usepackage{textcomp}
\usepackage{natbib}
\usepackage{hyperref}
\hypersetup{
    bookmarks=true,         % show bookmarks bar?
    unicode=false,          % non-Latin characters in Acrobat’s bookmarks
    pdftoolbar=true,        % show Acrobat’s toolbar?
    pdfmenubar=true,        % show Acrobat’s menu?
    pdffitwindow=false,     % window fit to page when opened
    pdfstartview={FitH},    % fits the width of the page to the window
    pdftitle={Smoderp manual},    % title
    pdfauthor={Kavka ...},     % author
    pdfsubject={Subject},   % subject of the document
    pdfcreator={Creator},   % creator of the document
    pdfproducer={Producer}, % producer of the document
    pdfkeywords={keyword1, key2, key3}, % list of keywords
    pdfnewwindow=true,      % links in new PDF window
    colorlinks=false,       % false: boxed links; true: colored links
    linkcolor=red,          % color of internal links (change box color with linkbordercolor)
    citecolor=green,        % color of links to bibliography
    filecolor=magenta,      % color of file links
    urlcolor=cyan           % color of external links
}
\usepackage[printonlyused]{acronym} %rejstik
% \usepackage{acronym} %rejstik
\makeatletter
\AtBeginDocument{%
  \renewcommand*{\AC@hyperlink}[2]{#2}%
}
\makeatother



%text
\usepackage{subcaption}
\usepackage{listings} % inserting code 

%tabulky
\usepackage{array}
\usepackage{tabulary}
\usepackage{multirow}
\usepackage{multicol}

%obrazky





% nuti obrazky aby nepretekali do dalsi sekce
\usepackage[section]{placeins}








%Lenka - nepouzivam
\usepackage{longtable}
\usepackage{siunitx}
\usepackage{rotating}
\usepackage[table,xcdraw]{xcolor}
\usepackage{booktabs}
\usepackage{url}
%\usepackage[pdftex,unicode,bookmarksnumbered,raiselinks=true]{hyperref}
% \usepackage{indentfirst}
\usepackage{fancyhdr}
\usepackage[font={footnotesize},labelfont=bf,justification=justified]{caption}
\usepackage{hhline}
\usepackage{colortbl}
\usepackage{array,graphicx}
\usepackage{placeins}

\usepackage{titlesec}



% \usepackage[table]{xcolor}
\usepackage{setspace}


% vetsi mezera mezi odstavci
\setlength{\parskip}{0.5em}


% nadefinovane tvary do flow chart diagramu
% ten jeden po postaven v ./graph/CZflowch
\usepackage{tikz}
\usetikzlibrary{shapes.geometric, arrows}
\tikzstyle{startstop} = [rectangle, rounded corners, minimum width=3cm, minimum height=1cm,text centered, draw=black, fill=red!30]
\tikzstyle{io} = [trapezium, trapezium left angle=70, trapezium right angle=110, minimum width=1cm, minimum height=1cm, text centered, draw=black, fill=blue!30]
\tikzstyle{arrow} = [thick,->,>=stealth]
\tikzstyle{decision} =  [diamond, minimum width=1cm, minimum height=1cm, text centered, text width=2cm, draw=black, fill=green!30, aspect=2]
\tikzstyle{process} = [rectangle, minimum width=3cm, minimum height=1cm, text centered, text width=3cm, draw=black, fill=orange!30]
\tikzstyle{guide} = [inner sep=0pt,minimum size=0mm]
\tikzstyle{line} = [thick]



\usepackage{dirtree}
\usepackage{forest}    % na dir tree ale obecnejsi uziti
% \usepackage{indentfirst}

% tohle je jen prikaz na delatni popisu rovnic 
% jeho pouziti he v kapitole pouzite vztahy treba
\newcommand{\jj}[2]{
   & \acs{#1} & je \acl{#1}#2 \\
}




% kolek okolo cisla, pouzito v tabulce popisujici toolbox
\newcommand*\circled[1]{\tikz[baseline=(char.base)]{
            \node[shape=circle,draw,inner sep=2pt] (char) {{\scriptsize\sffamily#1}};}}



% Název smodepu 
% 
\newcommand{\smod}{SMODERP2D\xspace}


% upraveni casti
\titleformat
{\part} % command
[display] % shape
{\bfseries\Huge} % format
{Část \ \thepart} % label
{5ex} % sep
{
%     \rule{\textwidth}{1pt}
%     \vspace{1ex}
%     \centering
} % before-code
[
\vspace{-2.5ex}%
\rule{\textwidth}{0.3pt}
] % after-code
 
 
%%%% upraveni sekce
% \titlespacing*{\section}
% {0pt}{7.5ex}{2.5ex} 
 


% pokud toto odkomentujes
% \newcommand*{\DEBUG}{}
% zobrazi se poznamky
% poznamky jsou oznaceny @@@
%   
% prikaz na delani poznamek
\newcommand{\pozn}[1]{
   \ifdefined\DEBUG
   {\color{red}(@@@ #1)}
   \fi
}

\begin{document}

  \thispagestyle{empty}
  

\begin{titlepage} % Suppresses headers and footers on the title page

	\centering % Centre everything on the title page
	
	\scshape % Use small caps for all text on the title page
	
	\vspace*{\baselineskip} % White space at the top of the page
	
	%------------------------------------------------
	%	Title
	%------------------------------------------------
	
	\rule{\textwidth}{0pt}\vspace*{-\baselineskip}\vspace*{2pt} % Thick horizontal rule
	\rule{\textwidth}{0.4pt} % Thin horizontal rule
	
	\vspace{0.75\baselineskip} % Whitespace above the title
	
	{\LARGE \smod\ - uživatelská příručka} % Title
	
	\vspace{0.75\baselineskip} % Whitespace below the title
	
	\rule{\textwidth}{0.4pt}\vspace*{-\baselineskip}\vspace{3.2pt} % Thin horizontal rule
	\rule{\textwidth}{0pt} % Thick horizontal rule
	
	\vspace{2\baselineskip} % Whitespace after the title block
	
	%------------------------------------------------
	%	Subtitle
	%------------------------------------------------
	
	Simulační Model povrchového ODtoku a ERozního Procesu 
	
	\vspace*{3\baselineskip} % Whitespace under the subtitle
	
	%------------------------------------------------
	%	Editor(s)
	%------------------------------------------------
	
	Edited By
	
	\vspace{0.5\baselineskip} % Whitespace before the editors
	
	{\scshape\Large Kavka \\ ... \\} % Editor list
	
	\vspace{0.5\baselineskip} % Whitespace below the editor list
	
	\textit{ČVUT} % Editor affiliation
	
	\vfill % Whitespace between editor names and publisher logo
	
	%------------------------------------------------
	%	Publisher
	%------------------------------------------------
% 	\includegraphics[width=0.5\textwidth]{./img/logo.png}
	
	
	
	\begin{verbatim}
    @ @ @   @       @     @ @     @ @ @     @ @ @ @  @ @ @    @ @ @
   @        @ @   @ @   @     @   @     @   @        @     @  @     @
   @        @   @   @  @       @  @      @  @        @     @  @     @
     @ @    @       @  @       @  @      @  @ @ @    @ @ @    @ @ @
         @  @       @  @       @  @      @  @        @   @    @
         @  @       @   @     @   @     @   @        @    @   @
    @ @ @   @       @     @ @     @ @ @     @ @ @ @  @     @  @

   \  \  /   / /    \   \  /   \  /    /     /        @ @ @   @ @ @
    \ _\/   /_/      \   \/     \/    /_____/        @     @  @     @
        \__/          \  /      _\___/                     @  @      @
            \____      \/      /                          @   @      @
                 \_____/______/                         @     @      @
                              \                       @       @     @
                               \____________________ @ @ @ @  @ @ @
 \end{verbatim}
 
 
 
% 	\plogo % Publisher logo
        
	
	\vspace{0.3\baselineskip} % Whitespace under the publisher logo
	
	2017 % Publication year
	
	{\large publisher} % Publisher

\end{titlepage}

%----------------------------------------------------------------------------------------






  \newpage
  \pagenumbering{roman}\setcounter{page}{1} % obsah a seznamy skratek, obrazku a tabulek rimska cislice od i
  \tableofcontents\addcontentsline{toc}{section}{Obsah}


  \newpage
  \section*{Seznam zkratek}\addcontentsline{toc}{subsection}{Seznam zkratek}
  \begin{multicols}{2}
  \input{./text_cz/akronymy}
  \end{multicols}

  \listoffigures\addcontentsline{toc}{subsection}{Seznam obrázků}
  \listoftables\addcontentsline{toc}{subsection}{Seznam tabulek}


  \newpage
  \pagenumbering{arabic}\setcounter{page}{1}% od uvodu do konce arabske cislice od 1
  \part*{Úvod}
  %!TEX ROOT = ../manual_CZ.tex


Dostává se Vám do ruky uživatelský manuál modelu \smod. Celý název modelu je: Simulační Model Povrchového Odtoku a Erozního Procesu. Tento model lze využít pro výpočet hydrologicko erozních procesů na jednotlivých pozemcích nebo na malých povodích. Výstupy z modelu jsou primárně určeny pro stanovení odtokových poměrů v ploše povodí a parametrů opatření pro snížení odtoku z povodí a erozního ohrožení zemědělské půdy. Model lze využít při navrhování komplexnějších soustav sběrných a odváděcích prvků nebo suchých nádrží a polderů. Jeho využití předpokládají jak současné metodiky, tak i technické normy a doporučené standardy.
Z hlediska kategorizace se jedná o fyzikálně založený plně distribuovaný dvourozměrný epizodní model. 
% 
Nově zavedené prostorové řešení (2D), které nahradilo dřívější profilovou verzi modelu, umožňuje komplexní řešení a náhled na celou řešenou lokalitu a přesnější popis zpravidla heterogenní morfologie zemského povrchu. 
% 
Přechod modelu na 2D řešení umožňuje zejména větší dostupnost potřebných dat a zvyšující se kapacita výpočetní techniky. 
% 
% Přesto, že dvourozměrné řešení je z hlediska vstupních dat a vnitřních procesů složitější, nicméně benefity distribuovaného řešení převažují. 
% Dostupnost vstupních dat v podrobném rozlišení se zlepšuje, stejně tak jako se zvyšuje výpočetní kapacita výpočetní techniky.
% 
% 
Vývoj modelu je podporován z veřejných prostředků a podílejí se na něm studenti a zaměstnanci Katedry hydromeliorací a krajinného inženýrství Fakulty stavební ČVUT v Praze.
Pro snazší orientaci je manuál rozdělen na dvou hlavních částí. 
% 
V první části jsou uvedeny zvolené výpočetní vztahy pro popis povrchového odtoku. 
% 
Druhá část je věnována popisu instalace a použití modelu v prostředí ArcGIS. Dále jsou zde podrobně popsána vstupní a výstupní data a stručně popsán tok programu. 
% Ve třetí části jsou ukázány výsledky z řečení konkrétní lokality.
Případné aktualizace, vzorová data, ukázky využití a další informace o modelu \smod jsou průběžně poskytovány na webových stránkách (\href{http://storm.fsv.cvut.cz/cinnost-katedry/volne-stazitelne-vysledky/smoderp/?lang=cz}{storm.fsv.cvut.cz/cinnost-katedry/volne-stazitelne-vysledky/smoderp/}).


\subsection*{Hydrologické modelování}\addcontentsline{toc}{subsection}{Hydrologické modelování}
Hydrologické modelování s využitím geografických informačních sytému (GIS) je oblast vyvíjející se od konce 70. let. Hydrologické modelu využívaly (a využívají) technologii operací s geodaty a databázemi, které GIS umožňuje. Přesto jsou některé topologické operace pro hydrologické modely specifické (např. definice rozvodnice povodí, lokalizace/odstranění bezodtokových míst)~\citep{devantier1993}. V zásadě se rozvíjeli modelu povrchového odtoku založené na pravidelném rastru nebo nepravidelné trojúhelníkové síť~\citep{devantier1993}. Rovněž docházelo k vývoji celistvých~\footnote{povodí je rozděleno na sub-povodí která jsou řešena celistvě například metodou SCS křivek} nebo plně distribuovaných modelů~\citep{devantier1993}. S vývojem informační techniky docházelo k větší integraci GIS softwaru a hydrologických modelů. Na obrázku~\ref{fig:klasGISHyd} je ukázána klasifikace propojení GIS a hydrologických modelu na konci 90. let~\citep{sui1999}. Hydrologický model buďto pouze využívá některé funkcionality GIS softwaru (obrázek~\ref{fig:klasGISHyd}a). V opačném případě již vývojář GIS softwaru integruje do systému některé prvky hydrologických modelů (obrázek~\ref{fig:klasGISHyd}b). GIS software a hydrologické model si pouze vyměňují některá data (pomocí binárních nebo ascii souborů (obrázek~\ref{fig:klasGISHyd}c), nebo GIS software umožňuje ve svém prostření vložit uživatelem vytvořené skripty, která pak tvoří hydrologický model (obrázek~\ref{fig:klasGISHyd}b)\footnote{Toto je případ modelu SMODERP2D pokud je spouštěn v prostředí ArcGIS}~\citep{sui1999}. Autoři studie se k těmto postupů staví kriticky. Jejich základní výhradou je, že struktury GIS software vytvářejí bariery pro popis hydrologických dějů, které mají jistá specifika. Konceptualizace pomocí vrstev ať už rastrových  nebo vektorových nemusí vyjadřovat heterogenitu hydrologických dějů nebo užití stochastických modelů. GIS softwary často umožňují pouze limitovanou prací časové proměnlivými ději~\citep{sui1999}.

\begin{figure}
  \centering
  \includegraphics[width=\linewidth]{./img/klasifikaceGISHyd.png}
  \caption{Klasifikace integrace hydrologických modelu podle~\cite{sui1999}}
  \label{fig:klasGISHyd}
\end{figure}

Při popisu povrchového odtoku se nejčastěji využívá jedno ze zjednodušení Saint-Venantových rovnic ať už kinematickou (ref) nebo difuzní vlnou (ref). 

% 
% 
  \newpage
  \part{Popis řešení}\label{cast:1}
  \input{./text_cz/mat_a_met/metody}
% 
% 

  \FloatBarrier
  \newpage
  \part{Použití modelu}\label{cast:2}
  \input{./text_cz/pouziti/postup}


  \FloatBarrier
  \appendix
  \section{Příloha: doplňující tabulky a grafy}\label{sec:priloha}
  \input{./text_cz/prilohy/priloha_1}

%   \appendix
  \FloatBarrier
  \section{Příloha: další výstupy}\label{sec:priloha2}
  \input{./text_cz/prilohy/priloha_2}


  \newpage

  \part*{Seznam použitých zdrojů}
  \bibliographystyle{agsm}
  \bibliography{bib/bib}

\end{document}



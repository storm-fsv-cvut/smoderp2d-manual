% zakomentovane je do tmp

\begin{table}[t]
 

 \centering
 \caption{Popis veličin  v {\tt.dat} souborech}
\label{tab:vystupydat}

% \begin{tabular}{p{4cm}lp{2cm}p{5cm}}
 \begin{tabular}{llp{0.5\textwidth}}
  \hline  \hline
 Název sloupce    & Jednotka    & Popis       \\ 
  \hline
%  \hline
%  Buňka s plošným odtokem:	 &&\\
 time[s]          &   $s$      &  Čas od začátku simulace          \\
 deltaTime[s]     &   $s$        &  Aktuální délka časového kroku  \\
 rainfall[m]      &  $m$         &  Srážková výška v aktuálním časovém kroku \\
%  sheetWaterLevel[m]       &  $m^3$  & Výška hladiny plošného odtoku \\
%  sheetFlow[m3/s]       &  $m^3s^{-1}$  & Průtok plošného odtoku  \\
%  sheetVolRunoff[m3]    &  $m^3$     & Odteklý objem plošného odtoku \\
%  sheetVolRest[m3]      &  $m^3$     & Objem zbytku vody po plošném odtoku \\
%  infiltration[m]         &  $m$      & Výška infiltrace v daném časovém kroku \\
%  surfaceRetention[m]    &  $m$      & Výška zadržené vody na povrchu v daném časovém kroku \\
%  callState                   &  -         & Typ odtoku na buňce (viz sekce~\ref{sec:statpopis})  \\
%  inflowVol[m3]   &   $m^3$ &  Celkový objem přítoku do buňky \\
 totalWaterLevel[m]	  &   $m$	&  Celková výška hladiny  \\ 
%  \hline
%  Pro soustředěný odtok &&\\ \hline 
%  rillWaterLevel[m]         &   $m$       &  Výška hladiny v buňce se soustředěným odtokem* \\
%  rillWidth[m]	       &   $m$ &  Šířka rýhy vzniklá soustředěným odtokem\\
%  rillFlow[m3/s]      &   $m^3s^{-1}$       &  Průtok v rýze soustředěného odtoku \\
%  rillVolRunoff[m3]   &   $m^3$  &   Objem soustředěného odtoku rýhou \\
%  rillVolRest[m3]  &  $m^3$ &   Objem zbytku vody po soustředěném odtoku rýhou  \\
 surfaceFlow[m3/s]   &  $m^3_{-1}$ & Celkový průtok (plošný + soustředěný)  \\
 surfaceVolRunoff[m3]   &   $m$  & Celkový odteklý objem (plošný + soustředěný) \\
%  rillInflowVol[m3] & $m$ &  @@@ toto tam chcem? to je V\_inflow cast co jde do ryhy, pridal jsem to tam jednou kdyz jsem hledal nejakou chybu...\\
%  ratio & $m$ &  Počet krácení časového kroku v rýhách @@@(je pro nas?)\\
%  sheetCourantCrit & $m$ &  Courantovo kritérium pro plošná odtok @@@(je pro nas?)\\
%  rillCourantCrit & $m$ & Courantovo kritérium pro soustředěný odtok @@@(je pro nas?) \\
%  nIter & $m$ &  Počet iterací pří výpočty daného výpočetního kroku @@@(to bych tam nechal, muže to napověděl jestli se tam neděje něco moc rychle, což může znamenat chybu v zadaných datech, třeba dát 600 mm do srážky místo 60 mm) \\
  \hline
   \hline
   \multicolumn{3}{p{\textwidth}}{*výška hladiny u soustředěného odtoku není výška skutečné výška hladiny v rýze, ale v nadkritická výška hladiny vztažená na celou plochu výpočetní buňky}
 \end{tabular}

\end{table}

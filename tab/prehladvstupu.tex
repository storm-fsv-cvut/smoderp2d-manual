
% 
\begin{sidewaystable}
% \begin{table}[]
\centering
\caption{Tabulka s přehledem vstulních dat modelu}
\label{tab:vstupy}
\small{
% \begin{tabular}{p{4cm}lp{2cm}p{5cm}}
\begin{tabular}{p{0.30\textwidth}lp{0.10\textwidth}p{0.30\textwidth}l}
\hline
Název                              & Typ dat                       & Povinný / volitelný & Poznámka                                                                                      & Více v kapitole                                                 \\ \hline \hline
digitální model terénu             & raster                        & Povinný           & Touto vrstvou se řídí i prostorová diskretizace.                                                 & \ref{sec:vstupdmt}                                           \\ 
prostorové rozložení půd           & vektor - polygony             & Povinný           & V atributové tabulce identifikátor typu půdy.                                               & \ref{sec:vstuppuda}                                          \\ 
prostorové rozložení využití území & vektor- polygony              & Povinný           & V atributové tabulce identifikátor využití území.                                           & \ref{sec:vstupvegetace} a \ref{sec:upravatabulkyparametru}   \\ 
srážková data                      & .txt soubor                   & Povinný           & Kumulativně zadaná srážka.                                                                       & \ref{sec:vstupsrazka}                                        \\ 
maximální časový krok              & reálné číslo                  & Povinný           & Model mění délku časového podle odtokových podmínek; doporučuje se 30 - 60 sekund.               & \ref{sec:vstupkrok}                                          \\ 
výstupní adrešář                   & text                          & Povinný           & Adresář k uložení výsledků (při spuštění výpočtu se obsah adresáře vymaže!).                            & \ref{sec:vstupadresar}                                       \\ 
bodové výstupy hydrogramů          & vektor - body                 & Volitelný         & Body pro výpis výsledků.                                                                   & \ref{sec:vstupbody}                                          \\ 
% typ výpočtu                        & text                          & Povinný           & Uživatel má na výběr: pouze plošní odtok, plošný i rýhový odtok, plošný, rýhový odtok i odtok hydrografickou sítí               & \ref{sec:vstupryhovy}          \\ 
% volba výcesměrného odtoku          & logická proměnná              & Povinný           & Jednosměrný (výchozí)  nebo vícesměnný odtok                                                           & \ref{sec:vstupvicesmerny}      \\ 
paramtry půdy a využití území           & tabulka                       & Povinný           & Tabulka parametrů půdy a využití území. Názvy sloupců mají definované označení. Hodnoty se spojí s vektorovými vrstvami.            & \ref{sec:upravatabulkyparametru}\\ 
hydrografická síť                  & vektor - linie                & Volitelný         & Prostorové rozložení hydrografické sítě. Atributová tabulka obsahuje identifikátor tvaru jednotlivých úseků.        & \ref{sec:vodnitoky}             \\ 
parametry úseků hydrografické sítě       & tabulka                       & Volitelný         & Tabulka parametrů jednotlivých úseků hydrografické sítě.                                                                        &  \ref{sec:vodnitoky}     \\ 
volba arcgis výstupů               & logická proměnná              & Povinný           & Výchozí formát výstupních rastrů je proprietární formát ERSI. Uživatel může zvolit textový formát ASCII.                       & --- \\ \hline
\end{tabular}
}
% \end{table}
\end{sidewaystable}
% \begin{itemize} \itemsep 0pt
% \item digitální model terénu
% \item shapefile půd
% \item shapefile využití území
% \item srážkový soubor
% \item časový krok výpočtu a celková doba simulace
% \item výstupní adresář
% \item bodová vrstva pro generování hydrogramů
% \item výstupní adresář
% \item typ výpočtu
% \item volba výcesměrného odtoku
% \item tabulka půd a vegetace a kód pro připojení
% \item shapefile hydrografické sítě
% \item tabulka vodních toků a kód pro připojení
% \item volitelné formy výstupů
% \end{itemize}

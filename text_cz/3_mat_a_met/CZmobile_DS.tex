%!TEX ROOT = ../mainCZ.tex
%\subsubsection{Mobilní dešťový simulátor}
\label{sss:ds_mobile}

Samotné zařízení mobilního dešťového simulátoru bylo podřízeno potřebám jednoduché ovladatelnosti a pro časté nasazení v terénu. Svou velikostí se řadí mezi nevětší mobilní dešťové simulátory používané v evropě. Rozměr zadešťované experimentální plochy 10x2 m. přesahuje mnohohonásobně většinu používaných zařízení \citep{Iserloh2013}  Sada devíti trysek v rozteči po 1,2 m je zapojena ve třech trojicích a je umístěna na rozkládacím rameni z příhradové.
Tato rozkládací konstrukce při trasportu umístěna na přívěsném vozíku za osobní automobil i s dalšími komponenty. Kromě vlastního ramene se zařízení skládá z nádrže na vodu o objemu 1000 litrů, řídicí jednotky s rozváděčem \label{kaitolas rovaděčem} a benzínového výkonného čerpadla. Dále jsou součástí stabilizační nohy celého ramene v rozloženém stavu, plechy sloužící k ohraničení ploch a plachta na celou konstrukci kuli větru.

stavba

schéma zapojení řídící jednotky atp

obrázky plochy

 
Klíčovýmy parametry pro měření pomocí DS jsou kinetická energie kapek, taková aby odpovídala přirozeným dešťům a rovnoměrnost postřiku na zadešťované ploše. Tyto parametry jsou dány typem trysek, jejich rozmístěním a tlakem v systému. Trysky jsou typu Spraying System WSQ 40. Změřené charakteristiky uvídí \cite{Strauss2000} a mají vzájemný odstup 1,2 m. Původní elektrické čerpadlo, napájené mobilní elektrocentrálou, bylo nahrazeno benzínovým čerpadlem o výkonu 120 l/min, který je dostatečný na vyrovnání ztrát v potrubí a zároveň dosažení až maximální teoretické intenzity 130 mm/hod. Při standardním měření se předpokládá intenzita 50–70 mm/hod.

Nejčastěji ve využívána délka měřené plochy 8 m, zbylý zadešťovaný prostor 2x2 m je využíván pro umístění srážkoměru a pro měření na menší zadešťované ploše (1x1 m), která je využita pro měření eroze působené výhradně plošným odtokem. Ohraničení plochy je pomocí plechové bariéry dole ukončenou koncentračním plechovým sběračem ve tvaru trychtýře, který slouží k odběru odtoku se splaveninami.
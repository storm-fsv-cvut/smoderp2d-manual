


Rozsah výstupních souborů, které jsou popsány v kapitole~\ref{kap:vystupy}, může být v určitých případech nedostatečný. Obecně se jedná o dva případy. V prvním případě je jejich rozsah nedostatečný z hlediska efektivního hledání chyb ve vstupních datech. V druhém případě vyžaduje samotná aplikace modelu  detailnější vhled do řešených procesů (například při vědeckých aplikacích). Proto existuje v modelu \smod možnost detailnějšího výpisů výsledků. 

Při běhu $preprocessingu$ jsou ukládány dočasné vrstvy do adresářů {\tt temp} a {\tt temp\_dp} (pokud jsou řešeny i úseky hydrografické sítě). Tyto adresáře jsou ve výchozím nastavení smazány na konci výpočtu. Dále je možné získávat detailnější informaci vypisovanou do hydrogramů a více výstupních rastrů s dalšími veličinami.

Pro získání detailnějších výsledků je třeba změnit jeden z parametrů modelu. Tento parametr je nejvíce využíván při vývoji modelu, a proto se zadává v jednom ze zdrojových skriptů modelu. 

V hlavním souboru v balíčku \smod ({\tt smoderp2d/main.py}) se volají metody ke načtení vstupních dat, $preprocessingu$ a k spuštění samotného výpočtu. V tomto souboru je metoda {\tt run()}, která je volána ve spouštěcím skriptu. V souboru {\tt smoderp2d/main.py} je $if$-konstrukt, který určí princip načtení vstupních dat na základě platformy, na které je model spuštěn:

    \begin{lstlisting}
    51  if platform.system() == "Linux" :
    52    from smoderp2d.src.main_classes.General import initLinux
    53    init = initLinux
    54  elif platform.system() == "Windows" :
    55    from smoderp2d.src.main_classes.General import initWin
    56    init = initWin
    57    sys.argv.append('#')               #  mfda
    58    sys.argv.append(False)             #  extra output
    59    sys.argv.append('outdata.save')    #  in data
    60    sys.argv.append('full')            #  castence nee v arcgis
    61    sys.argv.append(False)             #  debug print
    62    sys.argv.append('-')               #  print times
    63  else :
    64    from smoderp2d.src.main_classes.General import initNone
    65    init = initNone
    \end{lstlisting}
%     
Na řádku 58\footnote{číslování řádků se může lišit} je přidaná proměnná s hodnotou {\tt False} a s komentářem {\tt \#  extra output}. Pokud je tato proměnná změněna na {\tt True} nebudou se mazat po skončení výpočtu dočasné adresáře a hydrogramy i výstupní rastry budou doplněny o další výsledky. 




    
    
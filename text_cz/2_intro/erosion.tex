%!TEX ROOT = ../../main.tex
Water erosion is one of the most widespread forms of soil degradation. Reducing the erosion is one of many challenges worldwide and Europe \citep{LieveVan-Camp2004} or \citep{Boardman2006}. In particular, sediment transport from arable land into surface waters (streams, rivers, reservoirs) is one of the major problems of water management. Measuring and consequence modeling of the surface process are a necessary tool for the protection of the soil.

Two major surface processes influenced erosion (i) sheet flow with sheet erosion  and (ii) rill processes. Sheet flow energy is less than kinetic energy of the raindrops in sheet erosion \citep{Bryan2000}. Rate of erosion are influenced by vegetation cover that reduced impact of the rain energy. In the other hand rill erosion is generated by a concentrated flow and this process are more closely to stream processes  \cite{Gimenez2008, Govers2007}.

It is difficult to describe annual rate of soil erosion in the watershed over spatial and time scales. Long-term measurements and sufficient data base needed in order to investigate the response of erosion rates. Only abnormally high rainfall or an extreme event can produce main part of soil damage. Spatial scale of measuring are crutial as well. Many studies that focused to scaling (temporal and spatial) are published. For example \cite{Chaplot2012} deal with comparison of runoff and soil loss across the scales. In other studies \cite{Cerdan2002, Auerswald2009, BauerSGEM} the effect of the scales to erosion is evident.  

In a few available studies, the sediment yield from a hill slope or a catchment is likely to be less than the total sediment mobilised within it and estimated from plots  \cite{Walling1983}. Due to sedimentation, only a relatively small proportion of the detached and transported soil material reaches  the catchment outlet \cite{Beven2005, Verstraeten2001}.  Additional measurements and observations in different spatial scales in one place is  still required.

%!TEX ROOT = ../mainCZ.tex
%\subsubsection{Měření na žlabu}
%\label{sss:zlab}
Původní ovození odtokových parametrů MKWA byly stanoveny na základě vyhodnocení měření průtoků a výšek hladin na sklopném žlabu. Měření byla prováděna v Brně v roce 1984 \cite{Holy1984}. Systém spočíval v měření výšky hladiny proudící vody ve žlabu s různými druhy půd, které byly naneseny v tenké vrstvě na jeho dno a to při různých průtocích a sklonech. Díky tenké vrstvě půdy bylo možné uvažovat povrch bez vlivu infiltrace. Takto bylo provedeno měření na hladkém povrchu (plexisklo) a na třech typech půd (lehké, střední a těžké). Cílem měření bylo stanovit parametry vztahu mezi průtokem a výškou hladiny pro plošný povrchový odtok.
Tyto parametry byly určeny na čtyřech površích. Povrchem byla:

\begin{itemize}
\item lehká půda (částice jemného písku do velikosti 0,01 mm),
\item středně těžká půda (vzorek odebrán ve Velkých Žernosekách),
\item těžká půda (jílovité částice - spraš),
\item plexisklo (referenční povrch bez vzorku půdy).
\end{itemize}

Středně těžká půda, která byla odebrána z experimentálních ploch ve Velkých Žernosekách, obsahuje průměrně 22 \% zrn první kategorie, lze ji zatřídit jako hlinitopísčitou půdu. Půdy těžká a lehká byly vytvořeny uměle a obsahovaly pouze zrna dané kategorie. Průměrně lze těžké půdy zařadit podle Novákovy klasifikace jako jílovité půdy s průměrným obsahem jílnatých částic 75 \%. Lehkou půdu pak lze podle Novákovy klasifikace zatřídit jako písčitou s uvažovaným množstvím zrn první kategorie 10 %.
Prvním krokem při posuzování funkčnosti modelu bylo nejprve zopakováno a ověřeno vyhod-nocení výsledků měření na sklopném hydraulickém žlabu \cite{Holy1984}. Vyhodnocení jednotlivých měření uvádí tabulka (viz Tabulka 10).
Hlavním důvodem bylo ověřování ověření použitých jednotek. Původní statistické vyhodnocení bylo provedeno v jiných jednotkách, než ve byl napsán zdrojový kód.

%!TEX ROOT = ../main.tex

\begin{table}[htbp]
  \centering
  \caption{Original and corrected parameters for MKWA based on hydraulic trough}
    \label{tab:puvodni}%
%     \begin{tabular}{|ll|r|r|r|}
    \begin{tabular}{llccc}
    \hline 
    
	\multicolumn{2}{c}{Půdní druh} & \multicolumn{1}{c}{b} & \multicolumn{1}{c}{X} & \multicolumn{1}{c}{Y} \\
    \hline    
    \hline
	\multirow{2}[0]{*}{Písčitá} & původní & 1.8415 & 24.11 & 0.4869 \\
      & zpřesněné & 1.8614 & 25.47 & 0.4913 \\
          \hline
	\multirow{2}[0]{*}{Hlinitá} & původní & 1.748 & 28.64 & 0.541 \\
      & zpřesněné & 1.7362 & 29.46 & 0.5519 \\
          \hline
	\multirow{2}[0]{*}{Jílovitá} & původní & 1.5847 & 45.62 & 0.5614 \\
      & zpřesněné & 1.5847 & 47.52 & 0.5614 \\
          \hline
	\multirow{2}[0]{*}{Plexisklo} & původní & 1.6038 & 51.11 & 0.5471 \\
      & zpřesněné & 1.5928 & 49.81 & 0.5431 \\
          \hline
    \end{tabular}%

\end{table}%


%Výsledné hodnoty původní a nové jsou uvedeny v tabulce (viz Tabulka 10).
Ze srovnání vyplývá, že rozdíly hodnot jsou relativně malé. Tyto rozdíly jsou pravděpodobně způsobeny výpočtovými možnostmi nebo nutným zaokrouhlením. Pro další vývoj byly použity zpřesněné hodnoty.
Výsledky na hladkém povrchu je možné považovat za limitní a pro běžné půdy nepřekročitelné. Při porovnání mezi hladkým povrchem a jílem je patrné, že rozdíl není nijak významný, což je dáno velikostí jílových zrn.



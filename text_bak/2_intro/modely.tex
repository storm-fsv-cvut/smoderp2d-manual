%!TEX ROOT = ../main.tex
Computer based physical models can be used for erosion prediction over a wide range of conditions. To ensure model validity, simulation results must be compared with field measurements. Models can only work when they are applied to conditions. Due A desirable model should satisfy the requirements of universal acceptability; reliability; robustness in nature; ease in use with a minimum of data; and ability to take account of changes in land use, climate and conservation practices.

Many models was created and are constantly improved over the last twenty years. The specific conditions of formation, calibration and use of each models can't lead to universally valid model. General review article about models are in \cite{Pandey2016}. These fifty selected models essentially reflects the wide range of models including classification according they characterization.

Mathematical modeling is very important for describing erosion processes and for soil conservation. Generally, there are two types of models of soil erosion and surface runoff: (i) empirical models often USLE \citep{Wischmeier1978} or RUSLE \citep{Renard1991} based and (ii) physically based models. Parameterisation of the emperical models based on measurements often at standardised field plots in accordance with the estimation of USLE factors or on datasets from rainfall simulators \citep{Davidova2016}. 
Physically based models have for their goal to create a mathematical description of  process. Hydrolgical and hydraulical equations for watter balance and moving are base of approach. The models principally describe the processes of precipitation, infiltration, evapotranspiration, surface runoff, influence of vegetation. Loading, trasportation and sedimentation processes are in interconnection with surface water processes. Interconnections of precipitation, runoff, infiltration and soil erosion have become very important topics. A wide group of the models can be represented by WEPP \citep{Laflen1997}, KINEROS \citep{Woolhiser1989}, EROSION 2D/3D \citep{Schindewolf2012a} and model SMODERP that is object of this this article.
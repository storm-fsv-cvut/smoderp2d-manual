\documentclass[a4paper, 12pt, twoside]{article}
%Petr - pouzivam

%obecne
\usepackage[czech]{babel}
\usepackage[utf8]{inputenc}
\usepackage[IL2]{fontenc}
% \usepackage{charter}



\usepackage{xspace}
\usepackage{framed}
\usepackage{mathtools}
\usepackage{pdflscape}
\usepackage[top=3cm, left=3.5cm, right=2.5cm, bottom=3cm, headheight=15pt, includeheadfoot]{geometry}%rozměry stránky
\usepackage{textcomp}
\usepackage{natbib}
\usepackage{hyperref}
\hypersetup{
    bookmarks=true,         % show bookmarks bar?
    unicode=false,          % non-Latin characters in Acrobat’s bookmarks
    pdftoolbar=true,        % show Acrobat’s toolbar?
    pdfmenubar=true,        % show Acrobat’s menu?
    pdffitwindow=false,     % window fit to page when opened
    pdfstartview={FitH},    % fits the width of the page to the window
    pdftitle={Smoderp manual},    % title
    pdfauthor={Kavka ...},     % author
    pdfsubject={Subject},   % subject of the document
    pdfcreator={Creator},   % creator of the document
    pdfproducer={Producer}, % producer of the document
    pdfkeywords={keyword1, key2, key3}, % list of keywords
    pdfnewwindow=true,      % links in new PDF window
    colorlinks=false,       % false: boxed links; true: colored links
    linkcolor=red,          % color of internal links (change box color with linkbordercolor)
    citecolor=green,        % color of links to bibliography
    filecolor=magenta,      % color of file links
    urlcolor=cyan           % color of external links
}
\usepackage[printonlyused]{acronym} %rejstik
% \usepackage{acronym} %rejstik
\makeatletter
\AtBeginDocument{%
  \renewcommand*{\AC@hyperlink}[2]{#2}%
}
\makeatother



%text
\usepackage{subcaption}
\usepackage{listings} % inserting code 

%tabulky
\usepackage{array}
\usepackage{tabulary}
\usepackage{multirow}
\usepackage{multicol}

%obrazky





% nuti obrazky aby nepretekali do dalsi sekce
\usepackage[section]{placeins}








%Lenka - nepouzivam
\usepackage{longtable}
\usepackage{siunitx}
\usepackage{rotating}
\usepackage[table,xcdraw]{xcolor}
\usepackage{booktabs}
\usepackage{url}
%\usepackage[pdftex,unicode,bookmarksnumbered,raiselinks=true]{hyperref}
% \usepackage{indentfirst}
\usepackage{fancyhdr}
\usepackage[font={footnotesize},labelfont=bf,justification=justified]{caption}
\usepackage{hhline}
\usepackage{colortbl}
\usepackage{array,graphicx}
\usepackage{placeins}

\usepackage{titlesec}



% \usepackage[table]{xcolor}
\usepackage{setspace}


% vetsi mezera mezi odstavci
\setlength{\parskip}{0.5em}


% nadefinovane tvary do flow chart diagramu
% ten jeden po postaven v ./graph/CZflowch
\usepackage{tikz}
\usetikzlibrary{shapes.geometric, arrows}
\tikzstyle{startstop} = [rectangle, rounded corners, minimum width=3cm, minimum height=1cm,text centered, draw=black, fill=red!30]
\tikzstyle{io} = [trapezium, trapezium left angle=70, trapezium right angle=110, minimum width=1cm, minimum height=1cm, text centered, draw=black, fill=blue!30]
\tikzstyle{arrow} = [thick,->,>=stealth]
\tikzstyle{decision} =  [diamond, minimum width=1cm, minimum height=1cm, text centered, text width=2cm, draw=black, fill=green!30, aspect=2]
\tikzstyle{process} = [rectangle, minimum width=3cm, minimum height=1cm, text centered, text width=3cm, draw=black, fill=orange!30]
\tikzstyle{guide} = [inner sep=0pt,minimum size=0mm]
\tikzstyle{line} = [thick]



\usepackage{dirtree}
\usepackage{forest}    % na dir tree ale obecnejsi uziti
% \usepackage{indentfirst}

% tohle je jen prikaz na delatni popisu rovnic 
% jeho pouziti he v kapitole pouzite vztahy treba
\newcommand{\jj}[2]{
   & \acs{#1} & je \acl{#1}#2 \\
}




% kolek okolo cisla, pouzito v tabulce popisujici toolbox
\newcommand*\circled[1]{\tikz[baseline=(char.base)]{
            \node[shape=circle,draw,inner sep=2pt] (char) {{\scriptsize\sffamily#1}};}}



% Název smodepu 
% 
\newcommand{\smod}{SMODERP2D\xspace}


% upraveni casti
\titleformat
{\part} % command
[display] % shape
{\bfseries\Huge} % format
{Část \ \thepart} % label
{5ex} % sep
{
%     \rule{\textwidth}{1pt}
%     \vspace{1ex}
%     \centering
} % before-code
[
\vspace{-2.5ex}%
\rule{\textwidth}{0.3pt}
] % after-code
 
 
%%%% upraveni sekce
% \titlespacing*{\section}
% {0pt}{7.5ex}{2.5ex} 
 

%%%%%%%%%%%%   DĚLENÍ SLOV   %%%%%%%%%%%%%
\hyphenation{bio-logic-kých makro-fyta meteo-stanici geo-engi-nee-ring strmější}
%\DeclareUrlCommand\url{\def\UrlLeft{<}\def\UrlRight{>} \urlstyle{tt}}
\pagestyle{headings}
\setlength{\parskip}{8pt}
\setlength{\parindent}{16pt}

% ridke radkovani
\sloppy

% cesta k obrazkum
\graphicspath{ {img/} {graph/}}

% definice rotace bunky
\newcommand*\rot{\rotatebox[origin=c]{90}}

%http://tex.stackexchange.com/questions/63390/how-to-decrease-spacing-before-chapter-title
\makeatletter
\def\@makechapterhead#1{\vspace*{5\p@}
  {\parindent \z@ \raggedright \normalfont
    \ifnum \c@secnumdepth >\m@ne
        \huge\bfseries \@chapapp\space \thechapter
        \par\nobreak
        \vskip 20\p@
    \fi
    \interlinepenalty\@M
    \Huge \bfseries #1\par\nobreak
    \vskip 40\p@
  }}
\def\@makeschapterhead#1{%
  %%%%%\vspace*{50\p@}% %%% removed!
  {\parindent \z@ \raggedright
    \normalfont
    \interlinepenalty\@M
    \Huge \bfseries  #1\par\nobreak
    \vskip 40\p@
  }}
\makeatother

%http://tex.stackexchange.com/questions/53338/reducing-spacing-after-headings
\titlespacing\section{0pt}{12pt plus 4pt minus 2pt}{12pt plus 2pt minus 2pt}
\titlespacing\subsection{0pt}{12pt plus 2pt minus 2pt}{12pt plus 2pt minus 2pt}
\titlespacing\subsubsection{0pt}{6pt plus 4pt minus 2pt}{6pt plus 2pt minus 2pt}

%definice barev
\definecolor{tab_bg}{HTML}{C0C0C0}
\definecolor{tab_bg_light}{HTML}{EFEFEF}
\definecolor{tab_line}{HTML}{000000}
 

\begin{document}

\title{SMODERP - twenty years od development}
\author{Kavka}
%\maketitle 
Základní odvození vztahů povrchových procesů v modelu SMODERP vychází z rovnice kontinuity a rovnice pohybové na základě kinematického principu s využitím experimentálních měření. Základní bilanční rovnice lze zapsat ve tvaru.
\begin{eqnarray}
\frac{\mathrm{d}S}{\mathrm{d}t} = I_{t} - O_{t}
\end{eqnarray}
\ac{Itot}
\ac{O}
Obecnou rovnici je možné přepsat ve výškách hladin resp. objemů na dané ploše konkrétního elementu ve tvaru:
\begin{eqnarray} \label{bilancnirce}
Q nutno doplnit Kuba
\end{eqnarray}


\paragraph{\acs{ES}}  \mbox{} \\

Srážka je zdrojem a příčinou celého erozního procesu. Vzhledem k tomu, že se jedná o epizodní model je srážka zadávána v podobě konkrétní nebo návrhové srážky. Model počítá s vlivem intercepce, tedy že určitá část srážky bude zachycena rostlinami díky potenciální intercepci \acl{PotI}. Míra zachycení v každém výpočtovém čase je definována  pomocí poměrné plochy listové \acl{Lai} například \cite{Nevim}.

Část, která zůstane v časovém kroku na rostlinách v daném časovém kroku \acl{dT}, se dá vyjádřit jako násobek srážky a poměrné plochy listové:
\begin{eqnarray} \label{intercepce}
I_{veg} = P_{\Delta t}I_{LAI}
\end{eqnarray}

Z tohoto vztahu vyplývá, že efektivní srážku lze vyjádřit:

\begin{eqnarray} \label{intercepce}
I_{N} = P_{\Delta t}(1 - I_{LAI)}
\end{eqnarray}

Výše uvedený vztah platí až do vyčerpání potenciální intercepce, pak veškerá voda dopadá na půdní povrch.

\begin{eqnarray}
\sum_{\Delta t=0}^N I_{POT} \leq P_{I}
\end{eqnarray}
\ac{P}
\ac{O}
Po dosažení potenciální intercepce již nejsou rostliny schopny zadržovat další část srážky, a proto veškerá srážka je efektivní srážkou:

\paragraph{Infiltrace}  \mbox{} \\

V modelu je použita v současné době rovnice infiltrace podel Philipa \citep{PhillipXX} v následujícím tvaru:
\begin{eqnarray} \label{phillip}
I_{nf} = \frac{S_{i}T^{-1/2}}{2}+\frac{K_{i}}{\Delta t}
\end{eqnarray}

Philipova rovnice byla zvolena především z důvodu relativně malého počtu nutných vstupních parametrů. Autoři modelu si byli vědomi omezení použití Philipovy rovnice vyplývající z podmínek, za kterých byla odvozena.  Možné odchylky způsobené volbou této rovnice odpovídají odchylkám v heterogenitě půdy a kvalitě ostatních vstupů, na jejichž základě model pracuje.

V případě, že je infiltrace větší než srážka tak veškerá voda infiltruje do půdy

\paragraph{Plošný odtok} \label{rce_odtok}  \mbox{} \\

Rovnice plošného odtoku vychází z kinematického přístupu k řešení pohybové rovnice.
\begin{eqnarray}
q_{sur} = ah^{b}
\end{eqnarray}
Její odvození je provedeno na základě měření viz \ref{XXX} se zařazením plošného drsnostního součinitele podle Maninga pak přechází na tvar

\begin{eqnarray}
q_{sur} [m^{3}/s] = Ah_{sur}^{b} \Rightarrow \frac {1}{n} a h_{sur}^{X} i_{0}^{Y}
\end{eqnarray}

\textbf{ověřit sklon v \%}

Parametry \textit{a} a \textbf{b} jsou ovozeny na základě měření, viz kapitola \ref{XXX}. 
Z vyhodnocení vyplývá, že parametr b je závislý pouze na půdním druhu. Parametr a je závislý nejen na půdním druhu, ale také na sklonu svahu.
\\paragraph{Odtokové množství}
Z vypočteného průtoku, velikosti řešeného elementu a délky časového zle pak spočítat objem odtoku, které vstupuje do výsledné rovice bilance (\ref{bilancnirce}).

\begin{eqnarray}
O_{sur_{i,t}} [m^{3}] = \Delta t q_{sur}
\end{eqnarray}

Pro posouzení erozní ohroženosti a pro výpočet vzniku rýh je v každém elementu vypočítávána rychlost a tečné napětí

Výpočet rychlosti vychází zpětně z výpočtu průtoku za předpokladu a výšky hladiny za předpokladu, že se jedná a proudění vody o malé hloubce.

\begin{eqnarray}
v_{sur} =  \frac{q_{sur}}{h_{sur}} \label{eg:v}
\end{eqnarray}

Tečné napětí dále využívané v modelu pak uvažuje výpočet tak jak jej uvádí například \citep{Schwab1993}.
\begin{eqnarray}
\tau = \rho g h_{sur} i_{0} K \label{eg:tau}
\end{eqnarray}

Vypočítaná rychlost a tangenciální napětí jsou v případě posuzování erozní ohroženosti porovnávány s limitními hodnotami krajních nevymílacích rychlostí a tangenciálních napětí pro jednotlivé půdní druhy v závislosti na druhu vegetace \citep{Dyrova}
a jsou uvedeny v tabulce \ref{tabulkaDyrova}
.
V literatuře se setkáme i s odlišnými hodnotami. Například M. A. Velikanov stanovil krajní nevymílající rychlost pro půdy 0,24 $m s^{-1}$ \citep{cabik1963}, což je hodnota nižší, než kterou stanovila E. Dýrová.

\paragraph{Soustředěný odtok v rýhách}  \mbox{} \\

Výpočet soustředěného odtoku implementovaný do modelu SMODERP vychází z několika pědpokladů:
\begin{itemize}
\item zavedení stejných zjednodušujících předpokladů výpočtu proudění obdobně jako v případě výpočtu plošného odtoku. Nejedná o výpočet proudění v malé hloubce, ale předpokladem je že se v jednotlivých elemetech v relativně malých časových krocích jedná o rovnoměrné ustálené proudění. Při rovnoměrném proudění se předpokládá sklon dna $i_{0}$ rovný sklonu hladiny a shodná drsnost v celé délce elementu. Při stejném průřezu je hloubka konstantní, takže hladina probíhá rovnoběžně se dnem. Průtok Q $(m^{3}/s)$ je vyjádřen použitím Chézyho rovnice.
\item soustředěný odtok vniká v elementech, kde dojde k překročení kritických hodnot tečného napětí \ref{eg:tau} nebo rychlostí \ref{eg:v}. Objem vzniklé rýhy odpovídá nadkritickému množství vody.
\begin{eqnarray}
V_{rill}= V_{tot} - V_{crit} = (h_{sur} - h_{crit}) A_{el}
\end{eqnarray}

\item do výpočtu je zaveden jako zjednodušující předpoklad, že rýha má obdélníkový tvar v konstantním poměru hloubky a šířky tzv. \textit{rillrario}. Rozměry rýhy nejsou známy, protože rozměr rýhy se dynamicky mění v závislosti na množství vody během simulace.  Omočený obvod je pak možné převést podle následujícího vztahu \ref{eg:rill}. \\
\begin{eqnarray} \label{eg:rill}
R = \frac{S}{O} = \dfrac{h_{rill}b_{rill}}{b_{rill}+2h} = \dfrac{b_{rill}rill_{ratio}}{2rill_{ratio}+1}
\end{eqnarray}

\item v případě poklesu objemu vody v rýze si rýha zachovává svůj maximální tvar.
\end{itemize}

Pro výpočet je pak možné využít Chézyho rovnici v základním tavru:

\begin{eqnarray}\label{eq:chez}
Q = v A = C A \sqrt{Ri_{0}} 
\end{eqnarray} 
Rychlostní součinitel C $(m^{0.5}/s)$  závisí na drsnosti stěn koryta a na vegetačním krytu. Počítá se z empirických výrazů v modelu byl zvolena Manningova rovnice, kdy výpočet rychlosti proudění je pak:
\begin{eqnarray}
v = \frac{1}{n} R^{2/3} i_{0}^{1/2} 
\end{eqnarray} 

Dosazením do rovnice \ref{eq:chez} se dostane
\begin{eqnarray} \label{eg:maning}
Q =\frac A {1}{n} R_{h}^{2/3} i_{0}^{1/2}   
\end{eqnarray}
kde R (m) je hydraulický poloměr 

Poměr šířky a výšky je ve skriptu uložen jako parametr, který je možné v případě potřeby změnit. Objem rýhy je Stanovením poměru se získá jedna z neznámých b nebo y. Tu zbývající neznámou je možné získat díky znalosti objemu vody v jedné buňce v daném časovém kroku. Objem je dán ze vztahu
\begin{eqnarray}
V = byl 
\end{eqnarray} 
kde l (m) je délka rýhy v buňce. 

\subparagraph{Stanovení rozměrů rýhy}  \mbox{} \\
Při tvorbě rýh u simulace odtoku bylo uvažováno pro zjednodušení obdélníkové koryto o hloubce y, šířce b.\\

Průtočný průřez S se vypočte
\begin{eqnarray}
S = by
\end{eqnarray} 
Omočený obvod O
\begin{eqnarray}
O = b + 2y
\end{eqnarray} 

Rovnici \ref{eg:maning} je také možné zapsat ve tvaru 
\begin{eqnarray} \label{eq:total}
Q = \frac{\left(\frac{S}{O}\right)^{1/6}}{n} S \sqrt{\frac{S}{O} i_{0}}   
\end{eqnarray} 
Rychlost proudění v rýze se vypočte podle základního vztahu pro průtok
\begin{eqnarray} \label{eq:speed}
v = \frac{Q}{S} = \frac{Q * l}{V}   
\end{eqnarray}
kde V je objem vody, l je délka úseku v buňce.





\paragraph{Ppoznámka nebo to dát do diskuse k článku.}  \mbox{} \\
\begin{itemize}
\item Výsledný tvar blíží Maningově rovnici
\begin{eqnarray}
Q =\frac A {1}{n} R_{h}^{2/3} S^{1/2}
\end{eqnarray}
\item Přesněji pro tvar této rovnice pro plošný odtok, kdy se předpokládá proudění vody  o malé hloubce a tvar koryta je nahrazen jeho šířkou. Rovnice má pak tvar:
\begin{eqnarray}
Q =\frac {1}{n} h^{2/3} S^{1/2}
\end{eqnarray}
\item Že může být jiná rce infiltrace.
\item tvar rýhy - výzkum funkce?
\item jen jedna přímá rýha
\end{itemize}

\begin{acronym}
\setlength{\parskip}{0ex}
\setlength{\itemsep}{1ex}

\acro{a}[$a$]{parametr rovnice plošného odtoku [$?$]}
\acro{A}[$A$]{průtočná plocha  [$m^{2}$]}
\acro{b}[$b$]{parametr rovnice plošného odtoku [$?$]}
\acro{bhs}[$b$]{šířka dna příčného profilu hydrografické sítě [$m$]}
\acro{brill}[$b_{rill}$]{šířka rýhy [$m$]}
\acro{BrutoSR}[$B_{\Delta t}$]{srážka [$m$]}

\acro{CFL}[$CFL$]{Courant-Friedrich-Lewy podmínka}


\acro{D8}[$D8$]{jednosměrný odtokový algoritmus}

\acro{dT}[$\Delta t$]{časový krok [$s$]}
\acro{dTmax}[$\Delta t_{max}$]{maximální časový krok [$s$]}
\acro{dTmult}[$\Delta t_{mult}$]{multiplikátor časový krok [$-$]}
\acro{dX}[$\Delta x$]{prostorový krok [$m$]}
\acro{dS}[$\frac{\mathrm{d}S}{\mathrm{d}t}$]{změna zásoby [$m^3/s$]}

\acro{ES}[$ES$]{efektivní srážka [$m^3/s$]}
\acro{es}[$es$]{intenzita efektivní srážky [$m/s$]}
\acro{effVrst}[$l_{eff}$]{efektivní vrstevnice [$m$]}

\acro{hcrit}[$h_{crit}$]{výška hladiny [$m$]}
\acro{hrill}[$h^{rill}$]{hloubka rýhy [$m$]}
\acro{hsur}[$h^{sur}$]{výška hladiny na povrchu [$m$]}
\acro{HyVod}[$k$]{nasycená hydraulická vodivost [$m s^{-1}$]}
\acro{i}[$i$]{řešený element}

\acro{Inf}[$Inf$]{Infiltrace [$m^3/s$]}
\acro{inf}[$inf$]{intenzita infiltrace [$m/s$]}

% \acro{VItot}[$\dot{I_{tot}}$]{celkový přítok [$m^3/s$]}
\acro{Itot}[$I_{tot}$]{aktuální celkový [$m^3/s$]}

\acro{I}[$I$]{sklon [$-$]}

\acro{K}[$K$]{součinitel šířky (pro plošný povrchový odtok je K = 1)}
\acro{Ks}[$K_s$]{nasycená hydraulická vodivost [$m/s$]}

\acro{Lai}[$I_{LAI}$]{poměrná plocha listová [$-$]}
\acro{lrill}[$l_{rill}$]{délka rýhy [$m$]}
\acro{MKWA}[$MKWA$]{modifikovaná rovnice kinematické vlny}

\acro{mfda}[$mfda$]{vícesměrný odtokový algoritmus}
\acro{m}[$m$]{poměr sklonu svahů (pro obdélník je roven nule)}

\acro{n}[$n$]{manningův součinitel drsnosti}
\acro{NetoSR}[$I_{N}$]{efektivní srážka}

\acro{PS}[$PS$]{potenciální srážka [$m$]}

% \acro{VOtot}[$\dot{O_{tot}}$]{odtokové množství [$m^{3}$]}
\acro{Otot}[$O_{tot}$]{aktuální celkový odtok [$m^{3}/s$]}
% \acro{VOin}[$\dot{O^{in}}$]{objem přítoku ze sousední buňky [$m^{3}$]}
\acro{Oin}[$O^{in}$]{aktuální přítok ze sousedních buněk [$m^{3}/s$]}
\acro{oin}[$o^{in}$]{výška vtoku za čas [$m/s$]}
\acro{oinrill}[$o^{in}_{rill}$]{výška vtoku v rýze za čas [$m/s$]}
% \acro{VOout}[$\dot{O^{out}}$]{objem odtoku z buňky [$m^{3}$]}
\acro{Oout}[$O^{out}$]{aktuální odtok z buňky [$m^{3}/s$]}
\acro{oout}[$o^{out}$]{výška odtoku z buňky  za čas [$m/s$]}
\acro{ooutrill}[$o^{out}_{rill}$]{výška odtoku v rýze za čas [$m/s$]}


\acro{Q365}[$Q365$]{základní průtok [$m^3/s$]}

\acro{Orill}[$O_{rill}$]{objem odtoku - rýhový odtok [$m^{3}$]}
\acro{Osur}[$O_{sur}$]{objem odtoku - plošný odtok [$m^{3}$]}
\acro{O}[$O$]{omočený obvod [$m$]}
\acro{PotI}[$I_{POT}$]{potencionální intercepce [$m$]}
\acro{q}[$q$]{průtok [$m^{3}{s}^{-1}$]}
\acro{qrill}[$q_{rill}$]{průtok v rýhách [$m^{3}/s$]}
\acro{qsur}[$q_{sur}$]{specifický plošný průtok [$m^{2}/s$]}
% \acro{Qtot}[$q_{t}$]{celkový odtok}

\acro{qstream}[$q_{stream}$]{průtok v otevřeném korytě [$m^{3}/s$]}

\acro{Rrill}[$R_{rill}$]{hydraulický poloměr v rýze [m]}
\acro{Rstream}[$R_{stream}$]{hydraulický poloměr v otevřeném korytě [m]}

\acro{ret}[$ret$]{povrchová retence [m]}

\acro{R2}[$R^2$]{koeficient determinace}
\acro{ro}[$\rho$]{hustota [$kg/m^{3}$]}
\acro{rratio}[$rill_{ratio}$]{parametr tvaru rýhy [-]}
\acro{ratio}[$ratio$]{celočíselný faktor dělící časový krok při výpočtu rýhového odtoku}
\acro{slope}[$i_{0}$]{sklon[-]}
\acro{Sorb}[$S$]{sorptivita půdy [$m \sqrt{s}$]}
\acro{Sorbi}[$S_{i}$]{sorptivita půdy  v buňce $i$  [$m \sqrt{s}$]}
\acro{S}[$S$]{sorptivita půdy [$m \sqrt{s}$]}
\acro{t}[$t$]{časový krok[$s$]}
\acro{tausur}[$\tau_{sur}$]{tečné napětí [$Pa$]}
\acro{taucrit}[$\tau_{crit}$]{kritické tečné napětí [$Pa$]}

\acro{Vout}[$V_{out}$]{objem objem odtelkého [$m^{3}$]}

\acro{Vcrit}[$V_{crit}$]{objem vody do kritické hladiny [$m^{3}$]}
\acro{vrill}[$v_{rill}$]{rychlost proudění - rýhový odtok [$m/s$]}
\acro{Vrill}[$V_{rill}$]{objem vody v rýze v daném elementu [$m^{3}$]}
\acro{vsur}[$v_{sur}$]{rychlost proudění - plošný odtok [$m/s$]}
\acro{Vtot}[$V_{tot}$]{celkový objem vody v elementu [$m^{3}$]}
\acro{vstream}[$v_{stream}$]{rychlost proudění v úseku hydrografické sítě [$m/s$]}
\acro{vcrit}[$v_{crit}$]{kritická nevymílací rychlost [$m/s$]}
\acro{X}[$X$]{parametr rovnice plošného odtoku [$?$]}
\acro{Y}[$Y$]{parametr rovnice plošného odtoku [$?$]}
\acro{yrill}[$y_{rill}$]{parametr rovnice plošného odtoku [$m$]}
\acro{GIS}[$GIS$]{geografické informační systémy}
\acro{g}[$g$]{gravitační zrychlení [$m/s^{2}$]}

\acro{ij}[$i, j$]{souřadnice elementu - buňky}
\acro{MD8}[$MD8$]{Multiple Flow Direction Algorithm}

\acro{bunka}[$P$]{plocha buňky [$m^2$]}



\end{acronym}
%1to je \ac{MLS}
%2to je \acl{MLS}
%3to je \acs{MLS} 

%%!TEX ROOT = ../../main.tex
%\chapter*{}
% \section{Seznam použitých zdrojů}
\bibliographystyle{csplainnat}
% \addcontentsline{toc}{section}{Literatura}

\lhead[\sl{Literatura}]{}
\rhead[]{\sl{Literatura}}
%\bibliography{bib/erosion,bib/kavka,bib/models}
\bibliography{bib/bib}

\clearpage


%\printglossary[type=\acronymtype]
%\ac{HiRDLS} sdddsfsd
\end{document}

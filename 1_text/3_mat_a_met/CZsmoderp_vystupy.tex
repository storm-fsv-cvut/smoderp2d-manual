
\pozn{
\textbf{Zde dodelat}
\begin{itemize}
  \item popsat výstupy mimo temp
  \item popsat co jsou v temp
  \item popsat výstupy v určitých krocích
\end{itemize}
}



Výstupy modelu jsou uloženy do složky zadané mezi vstupními parametry (obsah složky je při spuštěné programu vymazán!). Kumulativní nebo maximální hodnoty na celém řešeném území jsou na konci výpočtu uloženy v rastrovém formátu (viz kapitola~\ref{sec:rastr}). Průnik polygonů prostorového rozlišení typu půd a využití území ve vektorovém formátu (viz kapitola~\ref{sec:vektor}). Pokud model \smod počítá i úseky hydrografické sítě, jsou kumulativní nebo maximální hodnoty veličin jednotlivých úseků vypsáný v atributové tabulce vektorové vrstvy úseků (viz kapitola~\ref{sec:vektor}), prostorové rozložení jednotlivých úseků je uloženo také jako jeden s rastrů (viz kapitola~\ref{sec:rastr}).  Volitelné výstupy hydrogramů v bodech ve formě časových řad jsou uloženy do textových souborů s příponou {\tt.dat} (viz kapitola~\ref{sec:hydrogramy}).  Jednotlivé výstupy jsou dále popsány podrobněji. 













\subsection{Rastrové výstupy}\label{sec:rastr}

V rastrech jsou uloženy vybrané veličiny na celém řešeném území. Jako rastrový formát lze zvolit proprietární ESRI formát nebo textový formát ASII. Přehled rastrových výstupních souborů je shrnut v tabulce~\ref{tab:vystupyrast}. Pokud jsou v modelu řešeny i úseky hydrografické sítě jsou buňky rastru ležící na úseku uloženy s hodnotou {\tt NoData} (výjimku tvoří 2 rastry popisující vlastnosti úseků, viz tabulka~\ref{tab:vystupyrast}).  

\begin{table}
 

 \centering
 \caption{Přehled rastrových výstupů}
\label{tab:vystupyrast}

% \begin{tabular}{p{4cm}lp{2cm}p{5cm}}
 \begin{tabular}{llp{0.5\textwidth}}
 \hline
 Název souboru    & Jednotka    & Popis       \\ 
 (ESRI nebo .acs)    &     &        \\ \hline \hline
 AreaRill       &   $m^2$      &  Plocha buňky, kterou porývá rýha \\
 CumInfiltL     &   $m$        & Kumulativní infiltrace \\
 CumRainL       &  $m$  &  Kumulativní srážka (bez intercepce a povrchové retence) \\
 CumVInL3       &  $m^3$  & Kumulativní objem přítoků do buňky  (plošný + rýhový) \\
 CumVOutL3       &  $m^3$  & Kumulativní objem odtoku z buňky \\
 CumVOutRillL3       &  $m^3$  & Kumulativní objem odtoku z buňky rýhou \\
 CumVRestL3      &  $m^3$  & Kumulativní zbytek po odtoku odtoku z buňky\\
 VRestEndRillL   &  - &  - \\
 FinalState    &  $NA$ & Typ odtoku buňky na konci výpočtu (viz sekce~\ref{sec:statpopis})\\
 HCrit            & $m$    &  kritický výška hladiny \\
 ShearStress   &   @@@ &  tečné napětí \\
 MaxQL3t\_1	  &   $m^3-s^{-1}$	&  Maximální plošný průtok v buňce  \\
 MaxQRillL3t\_1    &   $m^3-s^{-1}$	&  Maximální soustředěný průtok v buňce\\
 MaxVelovity	&   $ms^{-1}$	&  Maximální rychlost proudění v buňce (plošného či soustředěného odtoku) \\
 MaxWateL    &   $m$  &   Maximální výška hladiny plošného v buňce \\
 MaxWaterRillL   &   $m$  &   Maximální výška hladiny soustředěného odtoku v buňce \\
 toky   &  $NA$ &  Buňky mimo tok mají {\tt NODATA} hodnotu nebo id daného úseku toku.  \\
 Stream   &  $NA$ & doplnym  \\
 TotalBil   &   $m$  &  Bilance všech vstupů a výstupu z a do buňky  \\
 \end{tabular}

\end{table}



% 
% -rwxrwx--- 1 root vboxsf 48835 čec 10 23:16 AreaRill.asc*
% -rwxrwx--- 1 root vboxsf 48835 čec 10 23:16 CumInfiltL.asc*
% -rwxrwx--- 1 root vboxsf 48835 čec 10 23:16 CumRainL.asc*
% -rwxrwx--- 1 root vboxsf 48835 čec 10 23:16 CumVInL3.asc*
% -rwxrwx--- 1 root vboxsf 48835 čec 10 23:16 CumVOutL3.asc*
% -rwxrwx--- 1 root vboxsf 48835 čec 10 23:16 CumVOutRillL3.asc*
% -rwxrwx--- 1 root vboxsf 48835 čec 10 23:16 CumVRestL3.asc*
% -rwxrwx--- 1 root vboxsf 16698 čec 10 23:16 FinalState.asc*
% -rwxrwx--- 1 root vboxsf 67689 čec 10 23:16 HCrit.asc*
% -rwxrwx--- 1 root vboxsf 48835 čec 10 23:16 MaxQL3t_1.asc*
% -rwxrwx--- 1 root vboxsf 48835 čec 10 23:16 MaxQRillL3t_1.asc*
% -rwxrwx--- 1 root vboxsf 48835 čec 10 23:16 MaxVelovity.asc*
% -rwxrwx--- 1 root vboxsf 48835 čec 10 23:16 MaxWateL.asc*
% -rwxrwx--- 1 root vboxsf 48835 čec 10 23:16 MaxWaterRillL.asc*
% -rwxrwx--- 1 root vboxsf  5895 čec 10 23:16 point000.dat*
% -rwxrwx--- 1 root vboxsf  5895 čec 10 23:16 point001.dat*
% -rwxrwx--- 1 root vboxsf  5895 čec 10 23:16 point002.dat*
% -rwxrwx--- 1 root vboxsf  5894 čec 10 23:16 point003.dat*
% -rwxrwx--- 1 root vboxsf  5894 čec 10 23:16 point004.dat*
% -rwxrwx--- 1 root vboxsf  5895 čec 10 23:16 point005.dat*
% -rwxrwx--- 1 root vboxsf 18563 čen 13 13:04 point006.dat*
% -rwxrwx--- 1 root vboxsf  5895 čec 10 23:16 point007.dat*
% -rwxrwx--- 1 root vboxsf  5895 čec 10 23:16 point008.dat*
% -rwxrwx--- 1 root vboxsf  5895 čec 10 23:16 point009.dat*
% -rwxrwx--- 1 root vboxsf  5895 čec 10 23:16 point010.dat*
% -rwxrwx--- 1 root vboxsf  5895 čec 10 23:16 point011.dat*
% -rwxrwx--- 1 root vboxsf   408 čec 10 23:16 points.txt*
% -rwxrwx--- 1 root vboxsf 48835 čec 10 23:16 ShearStress.asc*
% -rwxrwx--- 1 root vboxsf 72730 čec 10 23:16 Stream.asc*
% -rwxrwx--- 1 root vboxsf   391 čen 13 13:04 stream.txt*
% -rwxrwx--- 1 root vboxsf 48835 čec 10 23:16 SurRet.asc*
% -rwxrwx--- 1 root vboxsf 16854 čec 10 23:16 toky.asc*
% -rwxrwx--- 1 root vboxsf 48835 čec 10 23:16 TotalBil.asc*
% -rwxrwx--- 1 root vboxsf 48835 čec 10 23:16 VRestEndRillL.asc*


% jsou ve textovém nebo rastrovém formátu.  













\subsection{Vektorové výstupy}\label{sec:vektor}

Výstupní data modelu ve vektorovém formátu jsou dva. Jedná se topograficky upravenou vrstvu \pozn{topograficka upraha = orientace linie atd...?} úseků hydrografické sítě ({\tt hydReach}), kde jsou do její atributové tabulky přidány kumulativní a maximální hodnoty vybraných veličin. Tyto veličiny jsou popsány v tabulce~\ref{tab:useky}. Druhým vektorovým výstupem je vrstva, která zobrazuje průnik prostorového rozložení typu půdy a využití území ({\tt interSoilLU}). Ukázka takové vrstvy je na obrázku~\ref{fig:pripravapar_vyrez}d. Tato vektorová vrstva slouží především ke kontrole správnosti přípravy vstupních dat či hledání chyb v nich. 


\begin{table}[h!]
 

 \centering
 \caption{Popis veličin  tabulky úseků hydrografické sítě}
\label{tab:useky}

% \begin{tabular}{p{4cm}lp{2cm}p{5cm}}
 \begin{tabular}{llp{0.5\textwidth}}
  \hline  \hline
 Název sloupce        & Jednotka     & Popis                                 \\ 
 \hline
 FUD              &   ---      &  Identifikátor přiřazeného úseku ({\it feature id})   \\
 V\_out\_cum           &  $m^3$         & Kumulativní objem odtoku            \\
 Q\_max [L3t-1]    &   $m^3s_{-1}$  & Maximální průtok                    \\
 timeQ\_max[s]          &   $s$          &  Čas dosažení maximálního průtoku    \\
 h\_max[m]        &  $m$  &  Maximální výška hladiny v úseku     \\
 timeh\_max[s]    &  $s$ &  Čas dosažení maximálního průtoku    \\
 Cumulatice\_inflow\_from\_field[L3]         &  $m^3$  &  Kumulativní přítok do úseku z jeho okolí \\
 Left\_after\_last\_time\_step[L3]            &  $m^3$  &  Objem v úseku po skončení výpočtu \\
 Out\_form\_domain[L3]              &  $m^3$     & Kumulativní odtok úseku z řešeného území (hodnota je nulová pokud úsek odtéká do jiného navazujícího úseku)   \\
 to\_reach              &  ---     & FID úseku do které daný úsek odtéká (hodnota -9999 vyjadřuje situaci, kdy úsek kříží hranici řešeného a odtéká tedy mimo toto území) \\
  \hline
   \hline
 \end{tabular}

\end{table}



% FID;V_out_cum [L^3];Q_max [L^3.t^{-1}];timeQ_max[s];h_max [L];timeh_max[s];
% Cumulatice_inflow_from_field[L^3];Left_after_last_time_step[L^3];Out_form_domain[L^3];to_reach













\subsection{Hydrogramy}\label{sec:hydrogramy}

Pokud jsou do vstupů zadány body pro výpis hydrogramů, vypíší se do testových souborů s příponou {\tt.dat}. Vypsané veličiny jsou závislé na typu odtokového procesu. Popis veličin při povrchovém odtoku je shrnut v tabulce~\ref{tab:vystupydat}. Pokud je v buňce úsek hydrografické sítě, vypisují se pouze hodnoty celého úseku. Názvy a význam veličin popisující úsek toku jsou popsány v tabulce~\ref{tab:vystupytokdat}. Model v současné verzi uvažuje, že pokud je v buňce úsek hydrografické sítě, zabírá úsek celou buňku, přesto že je eho šířka menší než šířka samotné buňky.  



% % zakomentovane je do tmp

\begin{table}[t]
 

 \centering
 \caption{Popis veličin  v {\tt.dat} souborech}
\label{tab:vystupydat}

% \begin{tabular}{p{4cm}lp{2cm}p{5cm}}
 \begin{tabular}{llp{0.5\textwidth}}
  \hline  \hline
 Název sloupce    & Jednotka    & Popis       \\ 
  \hline
%  \hline
%  Buňka s plošným odtokem:	 &&\\
 time[s]          &   $s$      &  Čas od začátku simulace          \\
 deltaTime[s]     &   $s$        &  Aktuální délka časového kroku  \\
 rainfall[m]      &  $m$         &  Srážková výška v aktuálním časovém kroku \\
%  sheetWaterLevel[m]       &  $m^3$  & Výška hladiny plošného odtoku \\
%  sheetFlow[m3/s]       &  $m^3s^{-1}$  & Průtok plošného odtoku  \\
%  sheetVolRunoff[m3]    &  $m^3$     & Odteklý objem plošného odtoku \\
%  sheetVolRest[m3]      &  $m^3$     & Objem zbytku vody po plošném odtoku \\
%  infiltration[m]         &  $m$      & Výška infiltrace v daném časovém kroku \\
%  surfaceRetention[m]    &  $m$      & Výška zadržené vody na povrchu v daném časovém kroku \\
%  callState                   &  -         & Typ odtoku na buňce (viz sekce~\ref{sec:statpopis})  \\
%  inflowVol[m3]   &   $m^3$ &  Celkový objem přítoku do buňky \\
 totalWaterLevel[m]	  &   $m$	&  Celková výška hladiny  \\ 
%  \hline
%  Pro soustředěný odtok &&\\ \hline 
%  rillWaterLevel[m]         &   $m$       &  Výška hladiny v buňce se soustředěným odtokem* \\
%  rillWidth[m]	       &   $m$ &  Šířka rýhy vzniklá soustředěným odtokem\\
%  rillFlow[m3/s]      &   $m^3s^{-1}$       &  Průtok v rýze soustředěného odtoku \\
%  rillVolRunoff[m3]   &   $m^3$  &   Objem soustředěného odtoku rýhou \\
%  rillVolRest[m3]  &  $m^3$ &   Objem zbytku vody po soustředěném odtoku rýhou  \\
 surfaceFlow[m3/s]   &  $m^3_{-1}$ & Celkový průtok (plošný + soustředěný)  \\
 surfaceVolRunoff[m3]   &   $m$  & Celkový odteklý objem (plošný + soustředěný) \\
%  rillInflowVol[m3] & $m$ &  @@@ toto tam chcem? to je V\_inflow cast co jde do ryhy, pridal jsem to tam jednou kdyz jsem hledal nejakou chybu...\\
%  ratio & $m$ &  Počet krácení časového kroku v rýhách @@@(je pro nas?)\\
%  sheetCourantCrit & $m$ &  Courantovo kritérium pro plošná odtok @@@(je pro nas?)\\
%  rillCourantCrit & $m$ & Courantovo kritérium pro soustředěný odtok @@@(je pro nas?) \\
%  nIter & $m$ &  Počet iterací pří výpočty daného výpočetního kroku @@@(to bych tam nechal, muže to napověděl jestli se tam neděje něco moc rychle, což může znamenat chybu v zadaných datech, třeba dát 600 mm do srážky místo 60 mm) \\
  \hline
   \hline
   \multicolumn{3}{p{\textwidth}}{*výška hladiny u soustředěného odtoku není výška skutečné výška hladiny v rýze, ale v nadkritická výška hladiny vztažená na celou plochu výpočetní buňky}
 \end{tabular}

\end{table}

% zakomentovane je do tmp

\begin{table}[t]
 

 \centering
 \caption{Popis veličin  v {\tt.dat} souborech}
\label{tab:vystupydat}

% \begin{tabular}{p{4cm}lp{2cm}p{5cm}}
 \begin{tabular}{llp{0.5\textwidth}}
  \hline  \hline
 Název sloupce    & Jednotka    & Popis       \\ 
  \hline
%  \hline
%  Buňka s plošným odtokem:	 &&\\
 time[s]          &   $s$      &  Čas od začátku simulace          \\
 deltaTime[s]     &   $s$        &  Aktuální délka časového kroku  \\
 rainfall[m]      &  $m$         &  Srážková výška v aktuálním časovém kroku \\
%  sheetWaterLevel[m]       &  $m^3$  & Výška hladiny plošného odtoku \\
%  sheetFlow[m3/s]       &  $m^3s^{-1}$  & Průtok plošného odtoku  \\
%  sheetVolRunoff[m3]    &  $m^3$     & Odteklý objem plošného odtoku \\
%  sheetVolRest[m3]      &  $m^3$     & Objem zbytku vody po plošném odtoku \\
%  infiltration[m]         &  $m$      & Výška infiltrace v daném časovém kroku \\
%  surfaceRetention[m]    &  $m$      & Výška zadržené vody na povrchu v daném časovém kroku \\
%  callState                   &  -         & Typ odtoku na buňce (viz sekce~\ref{sec:statpopis})  \\
%  inflowVol[m3]   &   $m^3$ &  Celkový objem přítoku do buňky \\
 totalWaterLevel[m]	  &   $m$	&  Celková výška hladiny  \\ 
%  \hline
%  Pro soustředěný odtok &&\\ \hline 
%  rillWaterLevel[m]         &   $m$       &  Výška hladiny v buňce se soustředěným odtokem* \\
%  rillWidth[m]	       &   $m$ &  Šířka rýhy vzniklá soustředěným odtokem\\
%  rillFlow[m3/s]      &   $m^3s^{-1}$       &  Průtok v rýze soustředěného odtoku \\
%  rillVolRunoff[m3]   &   $m^3$  &   Objem soustředěného odtoku rýhou \\
%  rillVolRest[m3]  &  $m^3$ &   Objem zbytku vody po soustředěném odtoku rýhou  \\
 surfaceFlow[m3/s]   &  $m^3_{-1}$ & Celkový průtok (plošný + soustředěný)  \\
 surfaceVolRunoff[m3]   &   $m$  & Celkový odteklý objem (plošný + soustředěný) \\
%  rillInflowVol[m3] & $m$ &  @@@ toto tam chcem? to je V\_inflow cast co jde do ryhy, pridal jsem to tam jednou kdyz jsem hledal nejakou chybu...\\
%  ratio & $m$ &  Počet krácení časového kroku v rýhách @@@(je pro nas?)\\
%  sheetCourantCrit & $m$ &  Courantovo kritérium pro plošná odtok @@@(je pro nas?)\\
%  rillCourantCrit & $m$ & Courantovo kritérium pro soustředěný odtok @@@(je pro nas?) \\
%  nIter & $m$ &  Počet iterací pří výpočty daného výpočetního kroku @@@(to bych tam nechal, muže to napověděl jestli se tam neděje něco moc rychle, což může znamenat chybu v zadaných datech, třeba dát 600 mm do srážky místo 60 mm) \\
  \hline
   \hline
   \multicolumn{3}{p{\textwidth}}{*výška hladiny u soustředěného odtoku není výška skutečné výška hladiny v rýze, ale v nadkritická výška hladiny vztažená na celou plochu výpočetní buňky}
 \end{tabular}

\end{table}


% % zakomentovane je do tmp



\begin{table}[h!]
 

 \centering
 \caption{Popis veličin  v {\tt.dat} souborech}
\label{tab:vystupytokdat}

% \begin{tabular}{p{4cm}lp{2cm}p{5cm}}
 \begin{tabular}{llp{0.5\textwidth}}
  \hline  \hline
 Název sloupce        & Jednotka     & Popis                                      \\ 
 \hline
 time[s]              &   $s$         &  Čas výpočetního kroku                    \\
 deltaTime[s]         &   $s$         &  Aktuální délka časového kroku            \\
 rainfall[m]          &  $m$          &  Srážková výška v aktuálním časovém kroku \\
 reachWaterLevel[m]        &  $m$          &  Výška hladiny plošného odtoku            \\
 reachFlow[m3/s]              &  $m^3s^{-1}$  &  Průtok plošného odtoku                   \\
 reachVolRunoff[m3]                  &  $m^3$     & Odteklý objem plošného odtoku     \\
%  reachVolInflow[m3]              &  $m^3$     & Suma přítoků z okolních buněk úseku v daném časovém kroku \\
%  reachVolRest[m3]         &  $m^3$     & Objem v úseku toku po odtoku      \\
  \hline
 \end{tabular}

\end{table}

% # Time[s];deltaTime[s];Rainfall[m];Waterlevel[m];V_runoff[m3];Q[m3/s];V_from_field[m3];V_rests_in_stream[m3]
% 
% 
% time[s]
% deltaTime[s]
% rainfall[m]
% reachWaterLevel[m]
% reachFlow[m3/s]
% reachVolRunoff[m3/s]
% reachVolInflow[m3]
% reachVolRest[m]
% zakomentovane je do tmp



\begin{table}[h!]
 

 \centering
 \caption{Popis veličin  v {\tt.dat} souborech}
\label{tab:vystupytokdat}

% \begin{tabular}{p{4cm}lp{2cm}p{5cm}}
 \begin{tabular}{llp{0.5\textwidth}}
  \hline  \hline
 Název sloupce        & Jednotka     & Popis                                      \\ 
 \hline
 time[s]              &   $s$         &  Čas výpočetního kroku                    \\
 deltaTime[s]         &   $s$         &  Aktuální délka časového kroku            \\
 rainfall[m]          &  $m$          &  Srážková výška v aktuálním časovém kroku \\
 reachWaterLevel[m]        &  $m$          &  Výška hladiny plošného odtoku            \\
 reachFlow[m3/s]              &  $m^3s^{-1}$  &  Průtok plošného odtoku                   \\
 reachVolRunoff[m3]                  &  $m^3$     & Odteklý objem plošného odtoku     \\
%  reachVolInflow[m3]              &  $m^3$     & Suma přítoků z okolních buněk úseku v daném časovém kroku \\
%  reachVolRest[m3]         &  $m^3$     & Objem v úseku toku po odtoku      \\
  \hline
 \end{tabular}

\end{table}

% # Time[s];deltaTime[s];Rainfall[m];Waterlevel[m];V_runoff[m3];Q[m3/s];V_from_field[m3];V_rests_in_stream[m3]
% 
% 
% time[s]
% deltaTime[s]
% rainfall[m]
% reachWaterLevel[m]
% reachFlow[m3/s]
% reachVolRunoff[m3/s]
% reachVolInflow[m3]
% reachVolRest[m]











% \subsection{State - typ průtoku na buňce}\label{sec:statpopis}
%   Jak bylo popsání no v kapitole~\ref{kap:tok} v modelu je možné řešit několik typů povrchového odtoku: plošný odtok, soustředěný odtoku a odtok hydrografickou sítí. Topografie hydrografické sítě je definována uživatelem. Vznik soustředěného odtoku je podmíněn překročením kritické výšky (popsáno v kapitole~\ref{sec:soustredenyodtok}). V programu jsou typy odtoku rozlišeny celočíselným identifikátorem označeným State, kde pokud State\\
% %   
%   \begin{tabular}{rcl}
%      =     &0&  dochází v buňce pouze k plošnému odtoku pokud \\
%      =     &1&  dochází v buňce k plošnému i soustředěnému odtoku  nebo pokud \\
%      =     &2&  @@@ plošný odtoku a rýha jen odtéká \\
%      && je to v teto verzi? Je to zajímavé pro uživatele? \\
%      $>$=  &1000&  je v buňce úsek hydrografické sítě. \\
%   \end{tabular}
% 
%   Identifikátor hydrografické sítě nemusí začínat číslem 1000 a nemusí být vzestupný (sestupný) u navazujících úseků. Tento identifikátor je v modelu definován jako 1000 + {\tt fid}, je tedy definován uživatelem nebo přiřazen použitým GIS softwarem. 
% 









%\section{Výstupní data} \label{section:vystupnidata}

%Po úspěšném ukončení modelu je do výstupního adresáře uloženo několik souborů. Každý z těchto souborů obsahuje hodnoty pro každou buňku rastru. Buňky, na kterých neprobíhal výpočet neobsahují žádné hodnoty, tedy NoData. Základní výstupy jsou uvedeny přímo ve zvoleném výstupním adresáři. Mimo hlavní výstupy jsou volitelně ukládány i dočasné výstupy sloužící pro případnou kontrolu. V podadresáři \textbf{temp} jsou dočasné soubory výpočtu v ploše a v podadresáři \textbf{temp_dp} jsou dočasné soubory vodních toků. \textbf{Dočasným výsledkům bude věnována jedna z dalších kapitol}






\pozn{
\textbf{tenhle seznam doplnit popisem o co tam jde a v jakých je to jednotkách}
\begin{itemize}
\item VRestEndRillL3.asc
\item TotalBil.asc
\item toky.asc
\item SurRet.asc
\item stream.shp
\item ShearStress.asc
\item body hydrogamů (*.dat) - průběh veličin pro jednotlivé body zadané v kapitole XXXX
\item MaxWaterRillL.asc
\item MaxWateL.asc
\item MaxVelovity.asc
%\item MaxQRillL3t_1.asc
%\item MaxQL3t_1.asc
%\item HCrit.asc
%\item FinalState.asc
\item CumVRestL3.asc
\item CumVOutRillL3.asc
\item CumVOutL3.asc
\item CumVInL3.asc
\item AreaRill.asc
\end{itemize}

\textbf{toto je origoš z DP}

\par Ne vždy se vytvoří všechny tyto výstupní soubory. Záleží na zvolených vstupních parametrech. Pokud uživatel nezadá žádnou bodovou vrstvu, nevytvoří se poslední textový soubor. 
V případě, že uživatel nezvolí možnost soustředěného odtoku, nevytvoří se rastry a shapefile související s tímto typem odtoku. Rastr soustředění odtoku se nevytvoří při nezvolení vícesměrného odtoku. 
Ostatní soubory se vytvoří pokaždé.  

\textbf{ z diplomky}

Výstupy se ukládají do adresáře nazvaného output. Cestu k němu si volí uživatel v rámci vstupních dat (viz kap. 2.3.1). Model prochází stále vývojem a dotýká se to i výstupních souborů. Princip ale zůstává stejný a jedná se spíše o úpravy zdrojového kódu zajištující lepší přehlednost a práci s kódem pro budoucí úpravy. Např. práce s vícerozměrnými maticemi a převedení všech výpočtů do základních (SI) jednotek. 
Výsledkem modelu jsou soubory (.shp, .rst, .txt, .dbf), které reprezentují parametry (Zajíček J., 2014):
hladina
Výstupem jsou hodnoty maximální výšky hladiny pro každou buňku. Jedná se tedy o rastrovou vrstvu vytvořenou porovnáváním hodnot výšek hladiny v každém časovém kroku. Uložena je nejvyšší hodnota. Výška hladiny v jednotlivých krocích je získána pomocí bilance přítoků a odtoků do buňky.  
průtok
Výstupem jsou hodnoty maximálního průtoku pro každou buňku. Obdobně jako u hladiny jsou porovnávány hodnoty v jednotlivých krocích a uložena maximální hodnota. Hodnoty průtoku v jednotlivých časových krocích jsou vypočteny pomocí metody kinematické vlny (teorie viz kap. 1.5.2).
infiltrace
Výstupem infiltrace jsou hodnoty v každé buňce, které jsou během doby běhu modelu postupně načítány až do vyčerpání infiltrační kapacity.
zbytkový objem
Zbytkovým objemem se rozumí objem, který v dané buňce v časovém kroku zůstal. V případě odtoku veškeré vody z rastru je rastr nulový. Matematicky je objem vyjádřen jako rozdíl celkového objemu v buňce (zbytkový objem z předchozího kroku a přítoky) a povrchového a soustředěného odtoku.
odtok
Výstup týkající se odtoku slouží pro konečnou bilanci (kontrolu) a testování. Jedná se o celkové množství, které z buňky odteklo za celou dobu běhu modelu. 
rychlost
Rastr rychlostí je výstupem sloužící k určení erozní ohroženosti. Porovnávány jsou hodnoty skutečných rychlostí s limitními nevymílacími rychlostmi (viz tab. č. 3 ).
napětí. 

Obdobou je rastr tečného napětí. Slouží k určení míst potencionálně nebezpečných. Hodnoty limitních hodnot tečného napětí jsou uvedeny ve stejné tabulce jako rychlosti průtok v rýze (viz tab. č. 3 ).

Průtok v rýze je rastrová vrstva znázorňující maximální průtok v rýze při soustředěném odtoku. Výstup je vytvořen jen při volbě typu výpočtu s uvažováním rýhového odtoku. Rýha vznikne pouze v buňkách, kde výška hladiny překročí hladinu kritickou. 
rychlost v rýze
Rastr obsahuje hodnoty maximální rychlosti v buňkách, kde je rýha vytvořena. Výpočet v rýhách probíhá odlišně oproti povrchovému odtoku. Jedná se o větší rychlosti, a proto na těchto buňkách probíhá výpočet za běžný časový krok 3x. V jiném případě by hrozilo, že výpočet nebude konvergovat.
souhrn

Final evalution.txt je textový soubor, který obsahuje souhrn zadaných vstupů a čas běhu modelu a bilanci vody. 
hydrogram
Point hydrographs.txt je textový soubor s hodnotami výšky hladiny, průtoku, napětí, rychlostí v bodech zadaných vstupní bodovou vrstvou. Soubor slouží k tvorbě hydrogramů v těchto bodech. Automaticky je k vrstvě přidán bod, ve kterém je hodnota flow acumulation nejvyšší.
Výstupem v současnosti je i řada dalších vrstev, které slouží ale spíše k tvorbě a testování modelu a pro samotného uživatele nejsou potřebné.	

}




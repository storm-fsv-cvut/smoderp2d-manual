%!TEX ROOT = ../mainCZ.tex
%\subsubsection{Měření na žlabu}
\label{sss:zlab}
Laboratorní dešťový simulátor (LabDS) byl konstruován od roku 1999 do roku 2002 a od té doby bylo provedeno přes dvěstěpadesát experimentů zemědělské půdy vystavený extrémním srážkovým událostem. Jedná se o tryskový dešťový simulátor typu \textit{Norton Ladder Rainfall Simulator} \cite{Norton2010} s kyvnými tryskami VeeJet s pádovou výškou kapky 2,43 m a průměrem kapky 2,3 mm. Zařízení umožňuje pro každou simulaci nastavit odlišné podmínky simulace, které se mohou lišit podélným sklonem půdního povrchu v rozmezí 0°–8°, intenzitou simulovaného deště v rozmezí 20–60~mm/hod. 
Plocha půdního vzorku je 4~x~0,9~m a hloubky 15~cm, přičemž dalších 5 cm tvoří spodní drenážní vrstva písku oddělená speciální mřížkou proti vyplavování sedimentu do spodního horizontu. Půdní vzorek je standartě proseta přes síta s velikostí ok 20x20 mm, aby došlo k odstranění rostlinných zbytků a větších frakcí. Po naplnění erozního kontejneru se půdní vzorek před začátkem simulací ponechává v klidu po dobu cca 4-6 týdnů, aby došlo k rovnoměrné konsolidaci celého vzorku.

% Table generated by Excel2LaTeX from sheet 'List1'
\begin{table}[htbp]
  \centering
  \caption{Přehled měření na labolatorním dešťovém simulátoru 2002 - 2016}
    \begin{tabular}{lcccc}
    \hline  \hline 
          &název / lokalita& rok & počet exp. & půdní druh \\
          \hline 
%          
    1     & Horoměřice & 2002/4 & 25    & jílovitohlinitá \\
%     \hline 
    2     & Třebsín I & 2004/6 & 22    & písčitohlinitá \\
%     \hline 
    3     & Neustupov & 2006/7 & 14    & hlinitopísčitá \\
%     \hline 
    4     & Klapý & 2007/8 & 25    & jílovitohlinitá \\
%     \hline 
    5     & Třebsín II & 2008/9 & 28    & písčitohlinitá \\
%     \hline 
    6     & Třebešice I & 2009/10 & 27    & hlinito-písčitá \\
%     \hline 
    7     & Třebešice II & 2010/11 & 36    & písčito-hlinitá \\
%     \hline 
    8     & Nučice & 2011/12 & 35    & hlinitá \\
%     \hline 
    9     & Všetaty I & 2012/13 & 24    & hlinitá \\
%     \hline 
    10    & Všetaty II & 2013/14 & 17    & hlinitá \\
%     \hline 
    11    & Třebešice III & 2014/15 & 22    & písčito-hlinitá \\
%     \hline 
    12    & Nové Strašecí & 2015/16 & 20    & hlinitá \\
    \hline  \hline 
    \end{tabular}%
  \label{tab:addlabel}%
\end{table}%


Při simulacích trvajících nejčastěji 60 minut se sleduje a zaznamenává průběh a rychlost povrchového odtoku, ze kterého se následně získává i průběh půdního smyvu, respektive celková ztráty půdy. Dalšími veličinami, které se zaznamenávají, jsou vlhkost půdního vzorku na začátku a konci simulace a kontrolní intenzity simulovaného deště po skončení měření [1]. 
Z hlediska objemu testovaných půdních vzorků se jedná o kvádr velikosti 4 x 0,9 m a hloubky 15 cm, přičemž dalších 5 cm tvoří spodní drenážní vrstva písku. Před naplněním kontejneru je půda navíc proseta přes síta s velikostí ok 20 x 20 mm, aby došlo k odstranění rostlinných zbytků a větších frakcí kameniva. Po naplnění erozního kontejneru se půdní vzorek před začátkem simulací ponechává v klidu po dobu cca 4-6 týdnů, aby došlo k rovnoměrné konsolidaci celého vzorku.

První kalibrace a následná validace odtokových parametrů byla provedena na výsledcích jednotlivých měření na laboratorním dešťovém simulátoru z let 2002 až 2008 \citep{KavkaDisertace}. Pro vlastní kalibraci a optimalizaci odtokových parametrů byl vytvořen model \textit{SmoderpMSExcel} a primárně byl určen pro jeden úsek svahu s konstantním sklonem a pro srážku s konstantní intenzitou.

Samotná kalibrace odtokových parametrů na základně naměřených hodnot probíhá ve dvou krocích. Nejprve byly určeny krajní hodnoty nasycené hydraulické vodivosti pro danou půdu vycházející z pedologického průzkumu. K těmto hodnotám byly přiřazeny hodnoty sorptivity. V těchto mezích jsou hledány hodnoty K a S tak, aby mezi měřenými a vypočtenými hodnotami byla maximální shoda. K nalezení optimálního řešení je používána metoda metody nejmenších čtverců. V dalším kroku je podobným způsobem pomocí lineární regrese metodou vyhodnocen i povrchový odtok. Měněnými parametry v tomto kroku řešitele jsou parametry rovnice \acs{MKWA}.

Výše popsaný postup byl aplikován na reprezentativní výběr měření z jednotlivých sad měření na různých půdách. Počet měření jednotlivých sad a počet z nich vybraných měření uvádí následující tabulka.

% Table generated by Excel2LaTeX from sheet 'List2'
\begin{table}[htbp]
  \centering
  \caption{Přehled experimentů použitých pro kalibraci odtokových parametrů}
  {\small
    \begin{tabular}{lllccc}
    \hline \hline
    Půdní druh & Lokalita & Propustnost & \multicolumn{1}{p{1.5cm}}{Obsah zrn první kategorie} & \multicolumn{1}{p{1.5cm}}{Počet měření} & \multicolumn{1}{p{2.0cm}}{Počet měření použitých pro kalibraci} \\
    \hline 
    Jílovitohlinitá  & Horoměřice & velmi nepropustná & 54    & 25    & 9 \\
%     \hline
    Písčitohlinitá  & Třebsín I & propustná & 29    & 21    & 9 \\
%     \hline
    Hlinitopísčitá  & Neustupov & málo propustná & 17    & 15    & 8 \\
%     \hline
    Jílovitohlinitá  & Klapý & velmi nepropustná & 54    & 16    & 6 \\
%     \hline
    Písčitohlinitá & Třebsín II & propustná & 29    & 30    & 8 \\
    \hline \hline
    \end{tabular}%
  }
  \label{tab:addlabel}%
\end{table}%


% \multicolumn{2}{c}{Půdní druh}
% \multicolumn{1}{p{1.5cm}}{Obsah zrn první kategorie}

U parametrů b, X bylo dostačující použití lineární regrese. Proložení složitější křivky nepřineslo výrazně těsnější vztah porovnáním hodnoty koeficientu determinace (R2). Pro parametr b se R2 rovná 0,87, pro parametr X se rovná 0,93. Pro parametr Y byl zvolen mocninný trend regresní křivky výsledným R2 = 0,83.

% Table generated by Excel2LaTeX from sheet 'List1'
\begin{table}[htbp]
  \centering
  \caption{Recalibrated parameters \citep{KavkaDisertace}}
    \begin{tabular}{lccc}
    \hline  \hline    
     soil type   &  \multicolumn{3}{c}{Runoff parameters}\\
     & \multicolumn{1}{c}{b} & \multicolumn{1}{c}{X} & \multicolumn{1}{c}{Y} \\
    \hline     
%     
    sand & 1.8165 & 23.3  & 0.4981 \\
    loamy sand & 1.7925 & 26.03 & 0.5202 \\
    sandy loam & 1.7685 & 28.75 & 0.5308 \\
    loam & 1.7385 & 32.16 & 0.5394 \\
    clay loam & 1.7025 & 36.26 & 0.5467 \\
    clayey & 1.6665 & 40.35 & 0.5521 \\
    clay   & 1.6185 & 45.8  & 0.5578 \\
    \hline   \hline    
    \end{tabular}%
  \label{tab:addlabel}%
\end{table}%
  

V rámci zpětného porovnání s dřívějšími verzemi bylo provedeno porovnání celkových výsledků pro čtyři verze modelu IV. I/11 – 96, 4.01, 5.01, 10.01 na svahu o délce 120 m se sklonem 6\%. U varianty modelu 10.01 byl upraven výpočet nejprve tak, aby se délka elementu rovnala délce celého svahu a výpočet probíhal stejně jako v předchozích verzích na svahu nerozděleném na elementy. Pak byl svah spočten pro variantu jednoduchý svah (délka elementu 10 m) a pro variantu násep (délka elementu 1 m).

% Table generated by Excel2LaTeX from sheet 'List1'
\begin{table}[htbp]
  \centering
  \caption{Porovnání výstupů verzí modelu 1996 až 2010}
    \begin{tabular}{lcccc}
    \hline   \hline 
      Id.    & Celk. objem & Max průtok & Max. výška hladiny & Max rychlost \\
          & [$m^{3}$]  & [$lsm^{-1}m^{-1}$] & [$mm$]  & [$ms^{-1}$] \\
    \hline
%   
    IV. I/11 – 96 120\_1 & 2.3   & 0.569 & 11.9  & 0.235 \\

    IV. I/11 – 96 120\_6 & 2.4   & 0.648 & 13.1  & 0.249 \\

    4.01 120\_1 & 2.3   & 0.44  &  N/A    &  N/A\\

    4.01 120\_6 & 2.32  & 0.51  &  N/A     &  N/A\\

    5.01 120\_1 & 2.2   & 1.6494 & 8.83  & 0.19 \\

    5.01 120\_6 & 2.29  & 2.2636 & 10.6  & 0.21 \\

    10.01 120\_1 & 2.06  & 2.04  & 10.24 & 0.199 \\

    10.01 char 1 & 2.315 & 3.1   & 13.05 & 0.239 \\

    10.01 char 3 & 2.347 & 3.42  & 13.77 & 0.248 \\
    \hline  \hline 
    \end{tabular}%
  \label{tab:addlabel}%
\end{table} 

Prostorová varianta modelu s podporou GIS aplikací začala být vyvíjena již v roce 2012  a jejím vývoji se pokračuje do teď. Nic méně, první problém, který se ukázal být zásadní byly nestandardní jednotky použité při kalibraci parametrů \acs{b}, \acs{X}, \acs{Y} rovnice \acs{MKWA}. Tyto nelineární a statisticky odvozené parametry nebylo možné jednoduše použít a bylo nutné přepočítat všechny vstupní hodnoty na jednotky SI a zopakovat kalibrační postup. Parametr \acs{b} zůstal po kalibraci zachován původní na rozdíl od parametrů \acs{X}, \acs{Y}. Nová kalibrace byla provedena na rozšířeném setu dat z dešťového simulátoru až do roku 2011. A pro verifikaci byla prvně využita data z terénních simulací pomocí dešťového simulátoru \citep{EGUDS}. Parametry odpovídající jednotkám SI jsou uvedeny v tabulce \ref{tab:Neuman_param}.

% Table generated by Excel2LaTeX from sheet 'List1'
\begin{table}[b!]
  \centering
  \caption{{Nově měřené hodnoty parametrů rovnice plošného odtoku \citep{Neumann15:232823}}}
    \begin{tabular}{lccc}
    \hline  \hline 
      soil type &     \multicolumn{3}{c}{Runoff parameters}\\
     & \multicolumn{1}{c}{b} & \multicolumn{1}{c}{X} & \multicolumn{1}{c}{Y} \\
	\hline      
    sand  & 1.8165 & 8.8133 & 0.3661 \\
    loamy sand & 1.7925 & 9.2043 & 0.4622 \\
    sandy loam & 1.7685 & 9.5953 & 0.5150 \\
    loamy & 1.7385 & 10.0841 & 0.5613 \\
    clay loam & 1.7025 & 10.6706 & 0.6028 \\
    clayey & 1.6665 & 11.2571 & 0.6358 \\
    clay  & 1.6185 & 12.0391 & 0.6717 \\
    \hline   \hline 
    \end{tabular}%
  \label{tab:Neuman_param}%
\end{table}%



Tyto parametry byly využity při stavbě prostorového řešení modelu Smoderp2D.
%!TEX ROOT = ../manual_CZ.tex


Dostává se Vám do ruky uživatelský manuál modelu \smod. Celý název modelu je: Simulační Model Povrchového Odtoku a Erozního Procesu. Tento model lze využít pro výpočet hydrologicko erozních procesů na jednotlivých pozemcích nebo na malých povodích. Výstupy z modelu jsou primárně určeny pro stanovení odtokových poměrů v ploše povodí a parametrů opatření pro snížení odtoku z povodí a erozního ohrožení zemědělské půdy. Model lze využít při navrhování komplexnějších soustav sběrných a odváděcích prvků nebo suchých nádrží a polderů. Jeho využití předpokládají jak současné metodiky, tak i technické normy a doporučené standardy.
Z hlediska kategorizace se jedná o fyzikálně založený plně distribuovaný dvourozměrný epizodní model. 
% 
Nově zavedené prostorové řešení (2D), které nahradilo dřívější profilovou verzi modelu, umožňuje komplexní řešení a náhled na celou řešenou lokalitu a přesnější popis zpravidla heterogenní morfologie zemského povrchu. 
% 
Přechod modelu na 2D řešení umožňuje zejména větší dostupnost potřebných dat a zvyšující se kapacita výpočetní techniky. 
% 
% Přesto, že dvourozměrné řešení je z hlediska vstupních dat a vnitřních procesů složitější, nicméně benefity distribuovaného řešení převažují. 
% Dostupnost vstupních dat v podrobném rozlišení se zlepšuje, stejně tak jako se zvyšuje výpočetní kapacita výpočetní techniky.
% 
% 
Vývoj modelu je podporován z veřejných prostředků a podílejí se na něm studenti a zaměstnanci Katedry hydromeliorací a krajinného inženýrství Fakulty stavební ČVUT v Praze.
Pro snazší orientaci je manuál rozdělen na dvou hlavních částí. 
% 
V první části jsou uvedeny zvolené výpočetní vztahy pro popis povrchového odtoku. 
% 
Druhá část je věnována popisu instalace a použití modelu v prostředí ArcGIS. Dále jsou zde podrobně popsána vstupní a výstupní data a stručně popsán tok programu. 
% Ve třetí části jsou ukázány výsledky z řečení konkrétní lokality.
Případné aktualizace, vzorová data, ukázky využití a další informace o modelu \smod jsou průběžně poskytovány na webových stránkách (\href{http://storm.fsv.cvut.cz/cinnost-katedry/volne-stazitelne-vysledky/smoderp/?lang=cz}{storm.fsv.cvut.cz/cinnost-katedry/volne-stazitelne-vysledky/smoderp/}).
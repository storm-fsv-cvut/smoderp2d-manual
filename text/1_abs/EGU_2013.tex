%!TEX ROOT = ../main.tex

The contribution presents a comparison of a runoff simulated by profile method (1D) and spatially distributed method (2D). Simulation model SMODERP is used for calculation and prediction of soil erosion and surface runoff from agricultural land. SMODERP is physically based model that includes the processes of infiltration (Phillips equation), surface runoff (kinematic wave based equation), surface retention, surface roughness and vegetation impact on runoff. 1D model was developed in past, new 2D model was developed in last two years.  The model is being developed at the Department of Irrigation, Drainage and Landscape Engineering, Civil Engineering Faculty, CTU in Prague. 
2D model was developed as a tool for widespread GIS software ArcGIS. The physical relations were implemented through Python script.  This script uses ArcGIS system tools for raster and vectors treatment of the inputs. Flow direction is calculated by Steepest Descent algorithm in the preliminary version of 2D model. More advanced multiple flow algorithm is planned in the next version.
Spatially distributed models enable to estimate not only surface runoff but also flow in the rills. Surface runoff is described in the model by kinematic wave equation. Equation uses Manning roughness coefficient for surface runoff.  Parameters for five different soil textures were calibrated on the set of forty measurements performed on the laboratory rainfall simulator. For modelling of the rills a specific sub model was created. This sub model uses Manning formula for flow estimation.
Numerical stability of the model is solved by Courant criterion. Spatial scale is fixed. Time step is dynamically changed depending on how flow is generated and developed.
SMODERP is meant to be used not only for the research purposes, but mainly for the engineering practice. We also present how the input data can be obtained based on available resources (soil maps and data, land use, terrain models, field research, etc.) and how can the model be used in the assessments of soil erosion risk and in designing of erosion control measures. 
The research has been supported by the research grants SGS SGS11/148/OHK1/3T/11 „Experimental Research on Rainfall-runoff and Erosion Processes“ and by Project No. TA02020647 „ Atlas EROZE - a modern tool for soil erosion assessment”.

